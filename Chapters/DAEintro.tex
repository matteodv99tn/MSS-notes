\chapter{Introduction to Algebraic Differential Equations}
	
	The \de{algebraic differential equations} \textbf{DAEs} can be regarded as a system of ordinary differential equations combined with \textit{general} algebraic equations; as example a DAE system is
	\[ \begin{cases}
		y' = f(x,y) \\
		y(a) = y_a \\
		f(x,y) = 0 \\
		x^2-y = 3
	\end{cases} \] 
	In general for ODEs and algebraic equations a lot of numerical methods have been implemented with consolidated theory regarding existence, stability... The problem is that when combining such theories in the algebraic differential equations, the numerical results that we might wanna retrieve are a \textit{nightmare} to compute.
	
	\paragraph{DAE with an example: simple pendulum} Considering the simple pendulum of a mass $m$ fixed by a bar of length $l$ to a pivot point; considering such center as the origin of a reference frame, the coordinates of the mass can be described as function of using the minimal number of coordinates (associated in this case to lagrangian coordinate $\theta$ as the angle between the bar and the vertical line) as
	\[ x= l\sin\theta \hspace{2cm} y = -l\cos\theta \]
	The idea is that this coordinates satisfy the constraint $x^2 + y^2 = l^2$, meaning that the mass $m$ can move only on the circle of radius $l$. Taking the velocities
	\[ \dot x = \frac {dx}{dt} = l\cos\theta \, \dot\theta \hspace{2cm} \dot y = \frac{dy}{dt} = l\sin\theta \,\dot\theta \]
	Computing the kinematic and potentials energy as
	\[ T = \frac m 2 \big(\dot x^2 + \dot y^2\big) = \frac m2 l^2 \dot \theta^2 \hspace{2cm} V = mgy = -mgl\cos\theta  \]
	With this we can build the lagrangian $\lag = T - V = \frac m 2 l^2 \dot \theta^2 + mg l\cos\theta$ and using the Euler-Lagrange equation (following the minimal action principle) the differential equation describing the motion is
	\[ \frac d{dt} \pd \lag {\dot\theta} - \pd \lag \theta = ml^2\ddot \theta + mgl\sin\theta = l\ddot \theta + g\sin\theta = 0 \]
	where the ordinary differential equation, in order to be solved/integrated, requires the initial conditions $\theta(0) = \theta_0$ and $\dot \theta(0) =\dot \theta_0$. Introducing $\omega = \dot \theta$ we can reduce the system of ODEs to the first order that can be numerically solved:
	\[ \begin{cases}
		\dot \theta = \omega \\
		l\dot \omega + g\sin\theta = 0 \\
		\theta(0)  = \theta_0 ,\quad \omega(0) = \omega_0 = \dot\theta_0
	\end{cases} \]
	Observe that we obtained the solution as an ordinary differential equation because we found the \textit{minimal} set of coordinates which describes the system.
	
	As alternative approach we could have used simpler independent ordinary differential equations and add some constraints; considering the independent mass $m$  described by the point $(x,y)$ in the plane and constrained to move on a circle of radius $l$, it's kinetic and potential energies are still $T = \frac m2 \big(\dot x^2 + \dot y^2\big)$ and $V = mgy$ (note that no transformation in terms of $\theta$ has been applied), then the lagrangian is
	\[ \lag = T-V = \frac m 2 \big(\dot x^2 + \dot y^2\big) - mgy \]
	The constraint is described by the equation $\phi(x,y) = x^2 + y^2 - l^2 = 0$. Adding to the constraint the least action principle stating that the functional $\mathcal A = \int_{t_0}^{t_1} \lag (x,y,\dot x,\dot y,t)\, dt$ we can find the \textit{stationary point} of the action $\mathcal A$ that's subject to $\phi(x,y) = 0$. Expanding the definition
	\[ \int_{t_0}^{t_1} \lag (x,y,\dot x,\dot y,t) - \lambda \phi(x,y)\, dt  \]
	we can build the hamiltonian $\H = \lag - \lambda \phi$ that, after the first variation, determines the system
	\[ \begin{cases}
		\frac d{dt} \pd \H{\dot x} - \pd \H x & = m\ddot x + \lambda x = 0 \\
		\frac d{dt} \pd \H{\dot y} - \pd \H y & = m\ddot y + \lambda y = -mg \\
		\phi(x,y) & = x^2 + y^2 - l^2= 0
	\end{cases} \]
	that is a \textbf{differential algebraic equation}; introducing $\dot x = u$ and $\dot y = v$ we can simplify to a differential algebraic equation
	\begin{equation} \label{eq:dae:daesys}
		\begin{cases}
			m \dot u + \lambda x = 0 \\
			m \dot v + \lambda y = -mg \\
			\dot x= u, \qquad \dot y = v\\
			x^2 + y^2 - l^2= 0
		\end{cases} 
	\end{equation}
	
	\paragraph{Introduction to numerical methods for DAEs} Considering the example of the pendulum, we can rewrite the differential equations as
	\[ \vector{\dot x \\ \dot y \\ \dot u \\ \dot v \\ 0 } = \vector{u \\ v \\ -\lambda x/m \\ -\lambda y / m - g \\x^2+y^2-l^2} \]
	Defining $\vett z = (x,y,u,b,\lambda)$, such relation is similar to the form $\dvett z = \vett F(t,\vett z)$. The main idea is in fact to \textbf{transform DAEs to ODEs}; note in fact that the last equation is the lonely one that's not already a ordinary differential equation: deriving it in time determines
	\[ \frac d{dt}\big(x^2 + y^2 - l^2\big) = 2 x\dot x + 2 y\dot y = 2x u + 2 yv  \]
	but stull we observe that the variable $\lambda$ is missing in the equation. Deriving one more time respect to $t$ we obtain
	\begin{equation} \label{eq:dae:secder} \begin{aligned}
		\frac{d}{dt}\big( 2x u + 2 yv \big) & = 2\dot x u + 2 x\dot u + 2\dot y v + 2 y \dot v = 2 u^2 + 2 v^2 - 2 x \frac{\lambda x}{m} - 2 y \left( \frac{\lambda y}m + g \right) \\
		& = 2\big( u^2 + v^2\big) - \frac{2\lambda}{m} \big(x^2 + y^2\big) - 2 y g
	\end{aligned} \end{equation}
	By substituting the different known relations for $\dot x,\dot y, \dot u, \dot v$ we so obtain a derivative that's function of $\lambda$, but not of $\dot \lambda$. Deriving one more time respect to the variable $t$
	\begin{align*}
		\frac{d}{dt}(\ref{eq:dae:secder}) & = 4\big(u\dot u + v\dot v\big) - \frac{4\lambda}{m} \big(x\dot x + y\dot y\big) - 2 \dot y g - \frac 2 m \dot \lambda \big(x^2 + y^2\big) \\
		& = 4 \left( - u \frac{\lambda  x}{m} - v \frac{\lambda y}{m} - vg \right) - \frac{4\lambda}{m} \big(xu + yv\big) - 2 vg - \frac 2 m \dot \lambda\big(x^2+y^2\big) = 0
	\end{align*}
	Solving for $\dot \lambda$ so gives
	\[ \dot \lambda = \frac{-4\lambda\big(xy+yv\big) -3 vmg }{x^2+y^2} \]
	We can so rewrite the differentia algebraic system in equation \ref{eq:dae:daesys} as a system of ODE only as
	\[ \begin{cases}
		\dot x = u \\ \dot y = v \\ \dot u = - \frac l m x \\ \dot v = - \frac \lambda m y - g \\ \dot \lambda = \frac{-4\lambda\big(xy+yv\big) -3 vmg }{x^2+y^2} 
	\end{cases} \]
	With such definition we can use numerical methods to solve the form $\dvett z = \vett F(t,\vett z)$. This formulations however introduces some problems:
	\begin{itemize}
		\item the initial condition on $\lambda$ is not set. This problem can be overcome considering that given $x$, the coordinate $y$ is constrained by $x^2 + y ^2 = l^2$. If we moreover know $\dot x = u$ then using the derivative of the constraint $2xy + 2yv = 0$, then also $v = \dot y$ is constrained. Using the second derivative of the constraint (equation \ref{eq:dae:secder}) we can finally solve for $\lambda_0$. We see that the initial conditions must satisfy the \textit{original} constraints and the \textit{hidden ones} determined by the derivatives of the algebraic equations. 
		
		\item another problem is that if we considered a constraint in the form
		\[ \phi(x,y) = x^2 + y^2 - l^2 +a + bt + ct^2 = 0 \]
		in order to obtain the ODE equivalent system we have to derive $\phi$ three times over time resulting in the cancellations of the polynomial  terms $a + bt+ ct^2$ (we in fact would have obtained the same ODE system).

	\end{itemize}

\section{Linear differential algebraic equations}
	Starting from the simplest cases of study, a generic differential algebraic equation can be written as
	\[ \begin{cases}
		\vett F(t,\vett y, \vett y') = 0 \\ \vett y(a) = \vett y_a
	\end{cases} \]
	with $\vett y(t) \in \mathds R^n$. We can say that the map $\vett F$ is \de{linear} in $y$ if it can be expressed as linear combination of $\vett y'$ in the form
	\begin{equation} \label{eq:dae:linear}
		\vett F(t,\vett y, \vett y') = \mat E(t,\vett y) \vett y' + \vett G(t,\vett y)
	\end{equation}
	Moreover the map $\vett F$ is linear in both $\vett y$ and $\vett y'$ if it can be regarded as
	\begin{equation} \label{eq:dae:linear2}
		\vett F(t,\vett y, \vett y') = \mat E(t) \vett y' + \mat A(t) y  - \vett C(t)
	\end{equation}
	where in general $\mat E,\mat A$ are matrices that can sometimes be singular. The map $\vett F$ is said \de{linear with constant coefficients} if it happens that the matrices $\mat E,\mat A$ are time independent.
	
	\begin{example}{: linear DAEs}
		An example of linear differential algebraic equation is the one that can be described as
		\[ \matrix{1 & t \\ t^2 & 2} \vector{ y_1' \\ y_2'} + \matrix{\sin t & \cos t \\ t^2 & 1} \vector{y_1 \\ y_2} - \vector{e^t \\ 1+t} \]
	\end{example}
	
	Observe that if the matrix $\vett E$ in non singular, expression \ref{eq:dae:linear} corresponds to a \textit{simple} ordinary differential equation: it can be in fact rewritten as
	\begin{equation}
		\vett y' = - \mat E^{-1}(t,\vett y) \vett G(t,\vett y) = \vett f(t,\vett y)
	\end{equation}
	For the moment we can assume that if $\mat E$ is singular the system is not an ODE. In this first part we will focus on linear differential algebraic equations in order to exploit linear algebra tools to ease the calculations.
	
	\paragraph{Jordan normal form} As a recall from the linear algebra, given a matrix $\mat B \in \mathds R^{n\times n}$ there exists always a non-singular matrix $\mat T \in \mathds R^{n\times n}$ such that 
	\[ \mat T^{-1} \mat B \mat T= \mat J\]
	where $\mat J$ is the \de{Jordan matrix form} defined as 
	\begin{equation}
		\mat J = \matrix{J_1 &&& 0 \\ & J_2 \\ && \ddots \\ 0 &&& J_m} \qquad \textrm{where } J_k = \matrix{ \lambda_k & 1 & & 0 \\ & \ddots & \ddots \\
		&& \ddots & 1 \\ 0 &&& \lambda_k }
	\end{equation}

	\paragraph{Regular pencil} Given the matrices $\mat B, \mat C \in \mathds R^{n\times n} $, the couple  $(\mat B,\mat C)$ is a \de{regular pencil} if 
	\[ f(\lambda) = \det\big( \mat B - \lambda \mat C \big) \neq 0 \]
	is not identically null, or equivalently if there exists a $\lambda$ such that $f(\lambda) \neq 0$. Considering as example the matrices 
	\[ \mat B = \matrix{1 & 1 \\ 0 & 0} \hspace{2cm} \mat C = \matrix{1 & 0 \\ 0 & 0 } \]
	is not a regular pencil, in fact the polynomial $f(\lambda) = \det(\mat B - \lambda \mat C) = \det \matrix{1-\lambda & 1 \\  0 & 0 } = 0$ always evaluates to zero. 
	
	\paragraph{Nilpotent matrix} A matrix $\mat B \in \mathds R^{n\times n}$ is \de{nilpotent} of order $p$ if
	\begin{equation}
		\mat B^p = 0 \quad\textrm{and} \quad \mat B^j \neq 0 \ \forall j < p 	
	\end{equation}
	where $\mat B^p$ is the product $\mat B \mat B\dots \mat B$ $p$ times. Considering as example
	\[ \mat B = \matrix{0 & 1 & 2 \\ 0 & 0 & 3 \\ 0 & 0 & 0} \qquad \mat B^2 = \matrix{ 0 & 0 & 3 \\ 0 & 0 & 0 \\ 0 & 0 & 0 } \qquad \mat B^3 = \matrix{ 0 & 0 & 0 \\ 0 & 0 & 0 \\ 0 & 0 & 0 } \]
	we have that such matrix $\mat B$ is nilpotent of order 3. Observe that if the matrix $\mat B$ is non-singular, than it can't be nilpotent.
	
	\paragraph{Kronecker normal form} If we consider two regular pencil matrices $(\mat B,\mat C) \in \mathds R^{n\times n}$, then there exists two non-singular matrices $\mat P, \mat Q \in \mathds R^{n\times n}$ such that 
	\begin{equation}
		\mat P \mat B \mat Q = \matrix{\mat N & 0 \\ 0 & \mat I} \qquad \textrm{and} \qquad \mat P \mat C \mat Q = \matrix{\mat I & 0 \\ 0 & \mat J}
	\end{equation}
	where $\mat N$ is a nilpotent matrix, $\mat I$ is the identity matrix and $\mat J$ is a Jordan normal form matrix. Considering that the \textit{blocks} $\mat N,\mat J$ can be empty, as extreme cases we have $\mat{PBQ} = \mat I$, $\mat{PCQ} = \mat J$, $\mat{PBQ} = \mat N$ and $\mat{PCQ} = \mat I$.
	
	\subsection{Usage of the Kronecker normal form}
		To ease the computation of linear differential algebraic equation, we can use the Kronecker normal form assuming that the couple of matrices  $(\mat E,\mat A)$ (equation \ref{eq:dae:linear2}) are a regular pencil (in order not to have an \textit{inconsistent} DAE).
		
		\begin{example}{: inconsistent DAE}
			Using the matrices defined used in the theory of regular pencil, we can build a DAE system of the form
			\[ \matrix{1 & 1 \\ 0 & 0} \vector{y_1' \\ y_2'} + \matrix{0 & 1 \\ 0 & 0 } \vector{y_1 \\ y_2} = \vector{t \\ 1} \]
			the associated system is
			\[ \begin{cases}
				y_1' + y_2' + y_2 = t \\ 0 = 1
			\end{cases} \]
			that's inconsistent.
		\end{example}
		 With such assumption we can compute the Kronecker normal form $\mat{PEQ} = \matrix{\mat N \\ & \mat I}$ and $\mat{PAQ} = \matrix{\mat I \\ & \mat J}$; premultiplying so equation \ref{eq:dae:linear2} by $\mat P$ results in $\mat{PE}\vett y' + \mat{PA}\vett y = \mat P \vett C$.
	\newcommand{\y}{\vett y} \newcommand{\z}{\vett z} \newcommand{\inv}{^{-1}}
		Performing the change of variable $\mat Q \z = \y$ (hence $\z = \mat Q\inv$) and observing that $\mat Q \z' = \y'$ we obtain the expression $\mat{PEQ} \z' + \mat{PAQ}\z = \mat P\vett C$ on top of which we can apply the Kronecker normal form:
		\begin{equation} \label{eq:dae:kroneckernormalform}
			\matrix{\mat N \\ & \mat I} \z' + \matrix{\mat I \\ & \mat J} \z = \mat P \vett C
		\end{equation}
		Splitting both vectors $\vett z = (\vett \alpha,\vett \beta)$ and $\mat P\vett C = (\vett d,\vett e)$ we can rewrite this expression as
		\[ \matrix{\mat N \\ & \mat I} \vector{\vett \alpha' \\ \vett \beta'} + \matrix{\mat I \\ & \mat J} \vector{\vett \alpha \\ \vett \beta} =\vector{\vett d \\ \vett e}  \]
		The associated linear system representation is
		\[ \begin{cases}
			i)&\mat N \vett \alpha ' + \vett \alpha = \vett d(t) \\
			ii)& \vett \beta' + \mat J \vett \beta = \vett e
		\end{cases} \]
		$ii)$ represent a \textit{standard} system of ordinary differential equations; the term $i)$ is instead more complex but we can invert the relation to obtain $\vett \alpha = \vett d - \mat N \vett \alpha'$. Observing that the $k$-th derivative in time of this expression evaluates to $\vett \alpha^{(k)} = \vett d^{(k)} - \mat N \vett \alpha^{k+1}$, we can substitute the derivatives $\vett \alpha^{(k)}$ determining
		\begin{align*}
			\vett \alpha & = \vett d - \mat N \vett \alpha' = \vett d - \mat N(\vett d' - \mat N \alpha'') = \vett d - \mat N \vett d' + \mat N^2 \vett \alpha'' = \vett d - \mat N \vett d' + \mat N^2 (\dots) = \dots \\
			& = \vett d - \mat N \vett d' + \mat N^2 \vett d'' - \mat N^3 \vett d''' + \mat N^4 \vett d^{(4)} - \mat N^5 \vett d^{(5)} + \dots
		\end{align*}
		This series is infinite, however being $\mat N$ a nilpotent matrix of order $p$ we have only that the first $p$ terms remains and so
		\begin{equation}
			\vett \alpha = \vett d - \mat N \vett d' + \mat N^2 \vett d'' +\dots  + (-1)^{p-1} \mat N^{p-1}\vett d^{(p-1)} + \cancel{(-1)^p \mat N^p \vett d^{(p)}} = \sum_{j=0}^{p-1} (-1)^j \mat N^j \vett d^{(j)}
		\end{equation}
		With this relation we determined $\vett \alpha$ without using the initial conditions (as was required for the DEA solution using the conversion to ODE), depends only on $\vett d$ (and it's derivatives up to the $p-1$ order). Also observe that $ii)$ is a \textit{regular} ODE, hence the initial values $\vett \beta(0)$ must be specified.
		
		The order of the nilpotency of the matrix $\mat N$ is the so called \de{index} of the differential algebraic equation (for the linear ones) and is a sort of \textit{measure} of the \textit{difficulty} of solving numerically the DAE. In particular if $p=0$ then what we have is a system of ordinary differential equation while if $p=n$ we have a set of only algebraic equations.
	
	\subsection{LU decomposition and Jacobi modification}
		In general the Kronecker normal form is \textit{hard} to compute; computationally we can use the \textit{simpler} \de{LU decomposition} in order to reduce the index of a differential algebraic equation, or better transform the DAE in an ordinary differential equation. 
		
		\paragraph{Recall on the LU decomposition} Considering $ \mat A \in \mathds R^{n\times n}$ a square matrix, then there exists 2 permutation matrices $\mat P,\mat Q$ such that
		\begin{equation}
			\mat{PA Q} = \mat{LU}
		\end{equation}	
		where $\mat L$ is a \textbf{lower triangular matrix} and $\mat U$ is an \textbf{upper} triangular one (in particular if the matrix $\mat A$ is singular only the first $m<n$ rows of $\mat U$ are non-zero and are still triangular, having a \textit{trapezoidal shape}).
		
		In the case that $\mat A$ is non-singular, then the algorithm can be reduced to the form $\mat{PA} = \mat{LU}$; alternatively we can use the \de{Echelon form} determined as $\mat{PA} = \mat{LUQ}^T = \mat L\tilde{\mat U}$. \vspace{3mm}
		
		A \textbf{permutation matrix} $S_{ij}$ are used to exchange the $i$-th and $j$-th row/column of a matrix $A$; in particular $S_{ij} A = \tilde A$ results in a swapping of the rows while $A S_{ij} = \tilde{\tilde A}$ is the exchange of the columns $i$ and $j$. Observe that $S_{ij} S_{ij} = \I$ results in the identity matrix and that permutations matrices are symmetric, in the sense that $S_{ij}^T = S_{ij}$.\\
	
		With that said if we consider a series of multiplication on permutation matrix we have that
		\[ P = S_{ij} S_{kl} \dots S_{mp} \qquad \Rightarrow \qquad P^T = S_{mp}^T \dots S_{kl}^T S_{ij}^T = S_{mp} \dots S_{kl} S_{ij} \]
		because we have that $P^TP=\I$. In general a permutation matrix if a series of product of exchanges collected in a single matrix $P$ such that $P^{-1} = P^T$.
		
		\paragraph{Jacobi modification} The main idea of the LU decomposition is to find $n-1$ matrices $\mat L_i$ such that $\mat L_{n-1} \dots\mat  L_2 \mat L_1 \mat{PAQ} = \mat U$ (where $\mat L = ( \mat L_{n-1}\dots \mat L_2\mat L_1)^{-1}$), where $\mat U$ is upper triangular and $L_i$ are all lower ones. The \de{Jacobi modification} leverage the same idea, but the permuted matrix $\mat{PAQ}$ is pre-multiplied by a series of Jordan normal matrices $\mat J_i$ resulting in a matrix of the form:
		\[ \mat J_n\dots \mat J_2\mat J_1 \mat{PAQ} = \matrix{\begin{array}{c|c}
			\mat I & \\ \hline 0 & 
		\end{array}} \]
		where the \textit{blank spaces} can be filled with non-zero elements.

\section{DAE index and index reduction}
	As already discussed, a linear differential algebraic equation can be regarded as $\mat E \vett y ' + \mat A \vett y = \vett c(t)$ (where we assume that the pair $(\mat E,\mat A)$ is a regular pencil), then with the Kronecker decomposition we had the formulation shown in equation \ref{eq:dae:kroneckernormalform} (page \pageref{eq:dae:kroneckernormalform}).  That allowed to re-state the original DAE problem into an algebraic part $\vett \alpha = \sum_{j=0}^{p-1} (-1)^j \mat N^j \vett d^{(j)}$ and an ordinary differential one $\vett \beta'(t) + \mat J \vett \beta(t) \vett e(t)$. Deriving the algebraic part evaluates to $\vett \alpha'(t) = \sum_{j=0}^{p-1} (-\mat N)^{j} \vett d^{(j+1)}(t)$, meaning that the Kronecker normal form allows to transform the original DAE problem into a system of ordinary differential equations:
	\begin{equation}
		\vector{\vett \alpha' \\ \vett \beta'} + \matrix{0 & 0 \\ 0 &  \mat J} \vector{ \vett \alpha \\ \vett \beta} = \vector{\vett e(t) \\ \sum_{j=0}^{p-1} (-\mat N)^{j} \vett d^{(j+1)}(t) }
	\end{equation}
	We can observe so that starting from a linear differential algebraic equation with constant coefficients  $\mat E \vett y ' + \mat A \vett y = \vett c(t)$ we have that after $p$ derivation (where $p$ is the nilpotency order of $\mat N$ in the Kronecker normal form) and \textit{some algebraic manipulation} we obtain an ordinary differential equation; as definition we say that the DAE has a \de{Kronecker index $p$}.
	
	\paragraph{Differential index} The minimum number of derivations of the system $\mat E \vett y ' + \mat A \vett y = \vett c(t)$ required to transform the DAE into an ODE is called \de{differential index}.
	
	Assuming the general expression $\vett F(\vett y, \vett y', t) = 0$, but also the derivatives $\frac d{dt}\vett F(\vett y,\vett y',t) = \pd{\vett F}{\vett y}\vett y' \pd{\vett F}{\vett y'} \vett y'' + \pd{\vett F}{t}$, $\frac{d^2}{dt^2}\vett F$ up to the $p$-th order $\frac{d^p}{dt^p}\vett F$, such values can be combined to obtain an expression in the form $\vett y' = \vett G(\vett y,t)$. The differential index is the minimum number of derivations $p$ required to transform $\vett F(\vett y,\vett y',t)= 0$ into $\vett g' = \vett G(\vett y,t)$.\\
	As special cases
	\begin{itemize}
		\item if the Kronecker normal form is  of type $\mat{PEQ} = \mat I$ and $\mat{PAQ} = \mat J$, then we have that
		\begin{align*}
			\mat{PE }\vett y' + \mat{PA} \vett y & =  \mat P \vett c \\
			\mat{PEQQ}^T \vett y' + \mat{PAQQ}^T \vett y & = \mat P \vett c \hspace{1.4cm} \leftarrow \quad \vett z = \mat Q^T \vett y \\
			\mat I \vett z' + \mat J \vett z & = \mat P \vett c \\ \Rightarrow \qquad \vett z' & = \mat P \vett c - \mat J \vett z
		\end{align*}
		This is a purely ordinary differential equation.
		\item alternatively if the Kronecker normal form is such that $\mat{PEQ} = \mat N$ and $\mat{PAQ} = \mat I$, then we have
		\begin{align*}
			\mat{PEQQ}^T \vett y' + \mat{PAQQ}^T \vett y & = \mat P \vett c \hspace{1.4cm} \leftarrow \quad \vett z = \mat Q^T \vett y \\
			\mat N \vett z' + \vett z & = \mat P \vett c = \vett d 
		\end{align*}
		Knowing that $\mat N$ is a nilpotent matrix of order $p$, recalling previous \textit{tricks we have that}
		\[ \vett z = \sum_{j=0}^{p-1} (-\mat N)^j \vett d^{(j)}  \]
		This is a purely algebraic equation.
	\end{itemize}
	
	\begin{example}{: DAE to ODE}
		Considering the differential algebraic equation
		\[ \begin{cases}
			\dot x_1 + \dot x_2 + x_1 = 1 \\
			\dot x_1 + \dot x_2 + x_1 + x_2 = t
		\end{cases} \]
		subtracting from the second equation the first one results in $x_2 = t-1$ that derived determines $\dot x_2 = 1$: we have so $\dot x_2$ expressed as an ordinary differential equation. Substituting this result in the first equation we have $\dot x_1 + 1 + x_1 = 0$, hence $\dot x_1 = -x_1$. After 1 derivation the resulting ODE is
		\[ \begin{cases}
			\dot x_1 = -x_1 \\ \dot x_2 = 1
		\end{cases} \]
		The differential index in this case is $1$ (because we only derived $\dot x_2 = t-1$ once).
	\end{example}
	
	\paragraph{Systematic index reduction algorithm} Given a linear DAE with constant coefficients $\mat E \vett y' + \mat A \vett y = \vett c$ (where usually $\mat E$ is singular), we can find two permutation matrices $\mat P, \mat Q$ in order to have the factorization $\mat{PEQ} = \mat{LU}$ where $\mat L$ is lower triangular and $\mat Q$ is \textit{upper trapezoidal} due to singularity of $E$. Knowing that $\mat Q^{-1} = \mat Q^T$, we also have that $\mat{PE} = \mat{LUQ}^T$: this last permutation $\mat Q^T$ corresponds simply to column commutations, meaning that we can regard
	\[ \mat{PE} = \mat{LUQ}^T = \mat{LM} \]
	where $\mat M$ is a matrix that in general is composed in two \textit{vertically stacked rectangular blocks}: the upper one that' generally non zero and the lower one identically null, hence $\mat M = \matrix{\begin{array}{c}
			\mat M_1 \neq 0 \\ \hline \mat M_2 =  0
	\end{array}}$. With this consideration we can pre-multiply the linear DAE by $\mat L^{-1} \mat P$ what we obtain is the formulation
	\begin{equation}
	\begin{split}
		\mat L^{-1} \mat{PE} \vett y' + \mat L^{-1} \mat{PA} \vett y & = \mat L^{-1} \mat P \vett c \\
		\mat M \vett y' + \mat N \vett y & = \vett d
 	\end{split}
	\end{equation}
	Splitting the matrices/vector $\mat M,\mat N, \vett d$ in order to \textit{match} the dimension of the blocks of $\mat M$, then we can consider the initial differential algebraic equations as made of
	\[ \begin{cases}
		\mat M_1 \vett y' + \mat N_1 = \vett d_1 \qquad & \textrm{: differential part} \\
		\mat N_2 \vett y = \vett d_2 & \textrm{: algebraic part}
	\end{cases} \]
	The easiest thing to do to reduce the DAE into a system of ordinary differential equation is to derive in time the algebraic part, that evaluates to $\mat N_2 \vett y' = \vett d_2'$: with that we obtain a \textit{pure} system of ordinary differential equations in the form
	\[ \begin{cases}
		\mat M_1 \vett y' + \mat N_1 = \vett d_1 \qquad & \textrm{: differential part} \\
		\mat N_2 \vett y' = \vett d_2' & \textrm{: algebraic part}
	\end{cases} \]
	This results in a \textit{new} differential algebraic system in the form $\mat E_1 \vett y' + \mat A_1 \vett y = \vett e_1$ where 
	\[ \mat E_1 = \matrix{\mat M_1 \\ \mat N_2} \qquad \mat A_1 = \matrix{\mat N_1 \\ 0} \qquad \vett e_1 = \vector{\vett d_1 \\ \vett d_2'} \]
	The so computed matrix $\mat E_1$ can still be singular, but we can apply this same algorithm until the obtained system is non-singular, hence solvable: the number of required derivations is the \de{index} of the differential algebraic equation.
	
	\begin{example}{: index computation and reduction} \label{ex:dae:indexreduction}
		Considering a differential algebraic system in the form
		\[ \begin{cases}
			\dot x_1 + \dot x_2 + \dot x_3 + x_1 = \sin t \\ 
			\dot x_1 + \dot x_2 + \dot x_3 + x_3 = t \\
			x_1 + x_3 = \cos t
		\end{cases} \]
		in order to perform the index reduction we can rewrite the system in a more \textit{compact} form in order to perform more easily the Gauss-Jordan solution of the system:
		\begin{align*}
			& \hspace{2.3mm} \begin{array}{ p{0.7cm} p{0.7cm} p{0.7cm}  p{0.7cm} p{0.7cm} p{0.7cm}  p{1cm} }
				$\dot x_1$ & $\dot x_2$ & $\dot x_3$ & $x_1$ & $x_2$ & $x_3$ & RHS
			\end{array} \\
			 & \left[ \begin{array}{ p{0.7cm} p{0.7cm} p{0.7cm} | p{0.7cm} p{0.7cm} p{0.7cm} | p{1cm} }
				1 & 1 & 1 & 1 & 0 & 0 & $\sin t$ \\
				1 & 1 & 1 & 0 & 0 & 1 & $t$ \\
				0 & 0 & 0 & 1 & 0 & 1 & $\cos t$
			\end{array} \right]
		\end{align*}
		Performing the transformation on the rows of such matrix $(2) \mapsto (2)-(1)$ we obtain the matrix
		\[ \matrix{ \begin{array}{ ccc | ccc | c }
				1 & 1 & 1 & 1 & 0 & 0 & \sin t \\
				0 & 0 & 0 & -1 & 0 & 1 & t-\sin t \\
				0 & 0 & 0 & 1 & 0 & 1 & \cos t
		\end{array} } \]
		Observing that the last two rows are identically zero for what concerns the entries in $\mat E$, then we can derive such algebraic part by \textit{shifting} to the left the block $\mat N_2$ and deriving in time the last column associated to the right hand side $\vett d$, obtaining
		\[ \matrix{ \begin{array}{ ccc | ccc | c }
			1 & 1 & 1 & 1 & 0 & 0 & \sin t \\
			-1 & 0 & 1 & 0 & 0 & 0 & 1-\cos t \\
			1 & 0 & 1 & 0 & 0 & 0 & -\sin t
		\end{array} } \]
		Continuing the Gauss-Jordan reduction
		\[ \xrightarrow[(3)\mapsto(3)-(1)]{(2)\mapsto(2)+(1)}\matrix{ \begin{array}{ ccc | ccc | c }
				1 & 1 & 1 & 1 & 0 & 0 & \sin t \\
				0 & 1 & 2 & 1 & 0 & 0 & 1-\cos t +\sin t \\
				0 & -1 & 0 & - 1 & 0 & 0 & -2\sin t 
		\end{array} } \]
		\[ \xrightarrow[(3)\mapsto(3)+(2)]{(1)\mapsto(1)-(2)}\matrix{ \begin{array}{ ccc | ccc | c }
			1 & 0 & - 1 & 0 & 0 & 0 & \cos t - 1 \\
			0 & 1 & 2 & 1 & 0 & 0 & 1-\cos t +\sin t \\
			0 & 0 & 2 & 0 & 0 & 0 & 1 - \cos t - \sin t
		\end{array} } \]
		\[ \xrightarrow[(2)\mapsto(2) - (3)]{(1)\mapsto(1)+\frac 12 (3)}\matrix{ \begin{array}{ ccc | ccc | c }
			1 & 0 & 0 & 0 & 0 & 0 & \frac{\cos t - \sin t - 1}{2} \\
			0 & 1 & 0 & 1 & 0 & 0 & 2 \sin t \\
			0 & 0 & 2 & 0 & 0 & 0 & 1 - \cos t - \sin t
		\end{array} } \]
		After all this reductions we obtained a non-singular matrix $\mat E$ determined just after one differentiation of the original problem (hence the index of the DAE is $p=1$) and the resulting ordinary differential equation is
		\[ \begin{cases}
			\dot x_1 = \frac 1 2 (\cos t - \sin t -1) \\ \dot x_2 = 2 \sin t - x_1 \\ \dot x_3 = \frac 1 2 (1-\cos t - \sin t)
		\end{cases} \]
	
	\end{example}
	
	\paragraph{When the algorithm fails} Considering now a differential algebraic system characterized by the equations
	\[ \begin{cases}
		x' + y' + z' + w' + x = t \\
		x' - x = t^2 \\
		y' - x = t \\
		x'+y' = 0
	\end{cases} \quad \leftrightarrow \quad \matrix{ \begin{array}{ cccc | cccc | c}
		1 & 1 & 1 & 1 & 1 & 0 & 0 & 0 & t \\
		1 & 0 & 0 & 0 & -1& 0 & 0 & 0 & t^2 \\
		1 & 0 & 0 & 0 & -1& 0 & 0 & 0 & t \\
		1 & 1 & 0 & 0 & 0 & 0 & 0 & 0 & 0
	\end{array} } \]
	after some steps in the Jordan-Gauss reduction for linear systems we obtain the form
	\[ \matrix{ \begin{array}{ cccc | cccc | c}
			1 & 0 & 0 & 0 & -1 & 0 & 0 & 0 & t^2 \\
			0 & -1& -1& -1& -2 & 0 & 0 & 0 & t^2-t \\
			0 & 0 & - 1 &-1 & -1 & 0 & 0 & 0 & -t \\
			0 & 0 & 0 & 0 & 0 & 0 & 0 & 0 & t - t^2
	\end{array} } \]
	Clearly the last row is a contradiction, in fact what it says is that $0= t-t^2$ that's not verified in general. This is due to the fact that the initial choice of the pair $(\mat A,\mat E)$ was \textbf{not} a \textbf{regular pencil}: we in fact have that
	\[ \det(\mat E - \lambda \mat A) = \left| \begin{array}{cccc}
		1-\lambda & 1 & 1 & 1 \\
		1+\lambda & 0 & 0 & 0 \\
		1+\lambda & 0 & 0 & 0 \\
		1 & 1 & 0 & 0 
	\end{array} \right| = - \left| \begin{array}{cccc}
		1+\lambda & 0 & 0  \\
		1+\lambda & 0 & 0 \\
		1 & 1 & 0  
	\end{array} \right| = 0 \]
	where the expansions used to compute the determinant are made along the last column on each sub-matrix.
	
	\paragraph{Linear DAE with non-constant coefficients} Considering the more general case of non-constant coefficient for linear differential algebraic equation, hence in the form $\mat E(t) \vett y' + \mat A \vett y = \vett c$, we might want to know if the condition on the regular pencil $(\mat A,\mat E)$ must still be required. Considering the simple system $\mat E(t) = \matrix{1 & t \\ 0 & 0}$ and $\mat A(t) = \matrix{0 & 0 \\ 1 & t}$ we can show that they are not regular pencil, in fact
	\[ \det(\mat E-\lambda \mat A) = \left| \begin{array}{cc}
		1 & t \\ 1-\lambda& t - \lambda
	\end{array} \right| = t-\lambda - t(1-\lambda) = 0\]
	However writing explicitly the differential algebraic equation we have
	\[ \begin{cases}
		x' + t y' = c_1 \qquad & \textrm{: differential part} \\
		x + ty = c_2 & \textrm{: algebraic part}
	\end{cases} \]
	Deriving in time the algebraic equation results in the system
	\[ \begin{cases}
		x' + t y' = c_1 \\ x' + t y' + y = c_2'
	\end{cases} \qquad \xrightarrow{(2) \mapsto (2)-(1)} \qquad \begin{cases}
		x' + t y' = c_1 \\
		y = c_2'- c_1
	\end{cases}\]
	Deriving one more time the second equation (that's algebraic) allows to compute a \textit{non-singular} system of ODEs characterized by solutions
	\[ y' = c_2'' - c_1' \qquad x' = c_1 - \big(c_2''-c_1'\big)t \]	
	The index of the original DAE is so 2.\\
	In general not being regular pencil for non-constant linear DAEs is not a problem; in fact the the system $\mat E(t)\vett x' + \mat A(t) \vett x = \vett c$, after some algebraic manipulation, can be reduced to a form
	\[ \matrix{\tilde{\mat E}_1(t) \\ 0 } \vett x' + \matrix{\tilde{\mat A}_1(t) \\ \tilde{\mat A}_2(t) } \vett x = \vector{\tilde{\vett c}_1 \\ \tilde{\vett c}_2 } \]
	Deriving the algebraic part this time doesn't mean just \textit{shifting} $\tilde{\mat A}_2(t)$ on the left to complex the matrix $\tilde{\mat E}$, but also introduces the derivative of the entries of $\tilde{\mat A_2}(t)$:
	\[ \matrix{\tilde{\mat E}_1(t) \\ \tilde{\mat A}_2(t) } \vett x' + \matrix{\tilde{\mat A}_1(t) \\ \tilde{\mat A}_2'(t) } \vett x = \vector{\tilde{\vett c}_1 \\ \tilde{\vett c}_2' } \]
	This makes the system in general non-singular, having the possibility always to have a solution.
	
	\paragraph{Tricks in the computation} Recalling example \ref{ex:dae:indexreduction}, after writing the linear system we described it in a \textit{unique} matrix containing both matrices $\mat E,\mat A$ and the right-hand side of the equation. This can be done for linear system, however we can note that the Gauss reduction is performed on the matrix $\mat E$ in order to obtain a form $\matrix{\mat I \\ & \mat 0}$. The general idea is so to compute the Gauss steps on the system
	\[ \matrix{ \ \mat E \ | \ \mat  I \ } \quad \xrightarrow{\textrm{Gauss  reduction}} \quad  \matrix{\ \mat T_m\dots \mat T_2 \mat T_1 \mat E \ | \ \mat T_m\dots \mat T_2 \mat T_1 \mat  I\ } =  \left[ \begin{array} {c  c |} 
		\mat I & \\ & \mat 0
	\end{array} \hspace{2mm} \mat T \hspace{2mm} \right] \]
	With such transformation matrix $\mat T$ computing $\mat {TE} \vett y' + \mat{TA} \vett y = \mat T \vett c$ results in a separation of the differential part from the algebraic one and we can apply the algorithm (by deriving the algebraic part and iterate).
	
	\begin{example}{: simple pendulum and index reduction}
		Recalling the simple pendulum described at the start of this chapter, we retrieved the differential algebraic equation describing the system as
		\[ \begin{cases}
			x' = u \\ y' = v \\ m y' + \lambda x = 0 \\ m v' + \lambda y = - mg \\ x^2 + y^2 - l^2 = 0
		\end{cases} \]
		Describing the time-dependent coordinates in the vector $\vett z' = (x,y,u,v,\lambda)$, the first thing to to is to re-state the problem in a form $\mat E(\vett z,t) \vett z' = \vett G(\vett z, t)$, so
		\[ \matrix{ 1  \\ & 1 \\ && m \\ &&& m \\ &&&& 0 } \vector{x' \\ y' \\ u' \\ v' \\ \lambda'} = \vector{u \\ v \\ -\lambda x \\ -\lambda y - mg \\ l^2 - x^2 - y^2} \]
		Exploiting the yet-described trick we consider the linear system
		\[ \matrix{\begin{array} {ccccc | ccccc}
			1 &&&&& 1 \\
			& 1 &&&&& 1 \\
			&& m &&&&& 1 \\
			&&& m &&&&& 1 \\
			&&&& 0 &&&&& 1 \\
		\end{array}} \]
		The first step of the index reduction is quite simple and consists in the multiplication by $\frac 1 m$ of both the $3^{rd}$ and $4^{th}$ rows, resulting in
		\[ \left[ \begin{array} {c  c |} 
			\mat I & \\ & \mat 0
		\end{array} \hspace{3mm} \mat T_1 \hspace{2mm} \right] = \matrix{\begin{array} {ccccc | ccccc}
			1 &&&&& 1 \\
			& 1 &&&&& 1 \\
			&& 1 &&&&& \frac 1m \\
			&&& 1 &&&&& \frac 1m \\
			&&&& 0 &&&&& 1 \\
		\end{array}}  \]
		Applying the transformation matrix $\mat T$ on the initial system determines so the form
		\[ \matrix{ 1  \\ & 1 \\ && 1 \\ &&& 1 \\ &&&& 0 } \vector{x' \\ y' \\ u' \\ v' \\ \lambda'} = \mat T_1 \vett G = \vector{u \\ v \\ - \frac \lambda m x \\ -\frac \lambda m  y - g \\ l^2 - x^2 - y^2} = \vett G_1 \]
		Deriving with respect to time the algebraic equation (that's only the last row) evaluates to $\frac{d}{dt}\big(l^2 - x^2 - y^2\big) = - 2xx' - 2y y'$; this determines the new system
		\[ \matrix{ 1  \\ & 1 \\ && 1 \\ &&& 1 \\ -2x & -2y & 0 & 0 & 0 } \vector{x' \\ y' \\ u' \\ v' \\ \lambda'} = \vector{u \\ v \\ - \frac \lambda m x \\ -\frac \lambda m  y - g \\ 0} \]
		
		We can apply the same method to reduce this system, by so determining a second transformation matrix $\mat T_2$ for this system using the Gauss reduction:
		\[ \matrix{\begin{array} {ccccc | ccccc}
				1   &&&&& 1 \\ 
				& 1 &&&&& 1 \\ 
				&& 1 &&&&& 1 \\ 
				&&& 1 &&&&& 1 \\ 
				-2x & -2y & 0 & 0 & 0 &&&&& 1
		\end{array}} \]
		\[ \xrightarrow{(5) \mapsto (5) + 2x (1) + 2 y(1)}\matrix{\begin{array} {ccccc | ccccc}
			1 &&&&& 1 \\ 
			& 1 &&&&& 1 \\ 
			&& 1 &&&&& 1 \\ 
			&&& 1 &&&&& 1 \\ 
			0 & 0 & 0 & 0 & 0 &2x& 2y & 0 & 0 & 1
		\end{array}} \]
		This new transformation matrix $\mat T_2$ determines a  new system
		\[ \matrix{ 1  \\ & 1 \\ && 1 \\ &&& 1 \\ &&&& 0 } \vector{x' \\ y' \\ u' \\ v' \\ \lambda'} = \mat T_2 \vett G_1 = \vector{u \\ v \\ - \frac \lambda m x \\ -\frac \lambda m  y - g \\ 2xu + 2yv} = \vett G_2 \]
		Deriving the algebraic part $\frac d{dt}(2xu + 2yv) = 2x'u +2x u' + 2y'v + 2yv'$ determines a new differential matrix $\mat E_3$ that can so be reduced using the Gauss method:
		\[ \matrix{\begin{array} {ccccc | ccccc}
				1   &&&&& 1 \\ 
				& 1 &&&&& 1 \\ 
				&& 1 &&&&& 1 \\ 
				&&& 1 &&&&& 1 \\ 
				2u & 2v & 2x & 2y & 0 &&&&& 1
		\end{array}} \]
		\[ \xrightarrow{(5) \mapsto (5) -2u (1) -2v (1) - 2x(3) - 2y(4)}\matrix{\begin{array} {ccccc | ccccc}
				1 &&&&& 1 \\ 
				& 1 &&&&& 1 \\ 
				&& 1 &&&&& 1 \\ 
				&&& 1 &&&&& 1 \\ 
				0 & 0 & 0 & 0 & 0 &-2u& -2v & -2x & -2y & 1
		\end{array}} \]
		determining
		\[ \matrix{ 1  \\ & 1 \\ && 1 \\ &&& 1 \\ &&&& 0 } \vector{x' \\ y' \\ u' \\ v' \\ \lambda'} = \mat T_3 \vett G_2 = \vector{u \\ v \\ - \frac \lambda m x \\ -\frac \lambda m  y - g \\ -2u^2 - 2v^2 +2 \frac \lambda m x^2 + 2 \frac \lambda m y^2 +2 yg} = \vett G_3 \]
		Deriving one more time the lonely algebraic equation results in
		\[ \frac d{dt} \left( -2u^2 - 2v^2 +2 \frac \lambda m x^2 + 2 \frac \lambda m y^2 +2 yg \right) = 2 \lambda' \frac{x^2 + y^2}{m} + 4\lambda \frac{x x' + yy'}{m} -4 uu' - 4 vv' +2y'g \]
		We so reduce the system
		\[ \matrix{\begin{array} {ccccc | ccccc}
				1   &&&&& 1 \\ 
				& 1 &&&&& 1 \\ 
				&& 1 &&&&& 1 \\ 
				&&& 1 &&&&& 1 \\ 
				4\frac \lambda m x & 4 \frac \lambda m y + 2g & -4u & -4v & 2\frac{x^2+y^2}{m} &&&&& 1
		\end{array}} \]
		\[ \xrightarrow{(5) \mapsto (5) - 4\frac \lambda m x  (1) - \left(4 \frac \lambda m y + 2g \right) (2) +4u (3) +4v(4)}\matrix{\begin{array} {ccccc | ccccc}
				1 &&&&& 1 \\ 
				& 1 &&&&& 1 \\ 
				&& 1 &&&&& 1 \\ 
				&&& 1 &&&&& 1 \\ 
				0 & 0 & 0 & 0 & 2\frac{x^2+y^2}{m} & - 4\frac \lambda m x & - 4 \frac \lambda m y - 2g &  4u & 4v & 1
		\end{array}} \]
		\[ \xrightarrow{(5) \mapsto \frac m{2(x^2+y^2)}(5)}\matrix{\begin{array} {ccccc | ccccc}
			1 &&&&& 1 \\ 
			& 1 &&&&& 1 \\ 
			&& 1 &&&&& 1 \\ 
			&&& 1 &&&&& 1 \\ 
			0 & 0 & 0 & 0 & 1 & - \frac{2 \lambda x}{x^2 + y^2} & - \frac{2\lambda y}{x^2+y^2} - \frac{mg}{x^2+y^2} & \frac{2mu}{x^2+y^2} & \frac{2mv }{x^2 + y^2} & \frac{m}{2(x^2+y^2)}
		\end{array}} \]
		Finally we have a transform matrix $\mat E$ that's non-singular and the ordinary differential equation originated from the initial DAE is so
		\[ \matrix{ 1  \\ & 1 \\ && 1 \\ &&& 1 \\ &&&& 1 } \vector{x' \\ y' \\ u' \\ v' \\ \lambda'} = \mat T_4 \vett G_3 = \vector{u \\ v \\ - \frac \lambda m x \\ -\frac \lambda m  y - g \\ \frac{-4\lambda(xy + yv) - 3mgv}{x^2 + y^2} } \]
	\end{example}

\subsection{Introduction of dummy variables}
	Considering the following differential algebraic equation retrieved by a fairly simple mechanical system that's in the form
	\begin{equation} \label{eq:dae:dummyinitial}
	\begin{cases}
		x_1' = u_1 \\ y_1' = v_1 \\ x_2' = u_2 \\
		u_1' = 2 \lambda_1 (x_1-x_2) + 2\lambda_2 x_1 \\
		v_1' = 2 y_1(\lambda_1 - \lambda_2) - y \\
		u_2' = 2 \lambda_2 (x_2 - x_1) \\
		x_1^2 + y_1^2 - 1 = 0 \\
		(x_1-x_2)^2 + y_1^2 - 1 = 0
	\end{cases}
	\end{equation}
	The difficult part for manually solving this problems lies in the implicit differentiation of the right hand sides of the ODEs: the three ones are quite simple (in fact $\frac d{dt}x_1' = u_1'$) while the others are more complex ($\frac d{dt}u_2' = 2\lambda_2'(x_2-x_1) + 2\lambda_2(x_2'-x_1')$) and so is not in general a good idea to perform the index reduction on such system (because each steps requires the differentiation in time, increasing the complexity of the calculations).\\
	The idea is so to introduce some \textbf{\textit{dummy variables}} in the form $\dot z$ that allows to rewrite the second three ODEs as
	\[ u_1' = \dot u_1 \qquad v_1' = \dot v_1 \qquad u_2' = \dot u_2 \] 
	subjected to the following algebraic constraints:
	\[ 0 = 2\lambda_1(x_1-x_2) + 2 \lambda_2 x_1 - \dot u_1 \qquad 0 = 2y_1(\lambda_1-\lambda_2) - y - \dot v_1 \qquad 0 =  2\lambda_2 (x_2-x_1) - \dot u_2  \]
	With this idea the initial differential algebraic equation can be regarded as
	\begin{equation} 
		\left\{ \begin{aligned}
			&\left. \begin{aligned}
				x_1' = u_1 \\ y_1' = v_1 \\ x_2' = u_2 \\ u_1' = \dot u_1 \\ v_1' = \dot v_1 \\ u_2' = \dot u_2
			\end{aligned} \hspace{42mm} \right\} &&\textrm{: simpler ODE part} \\
			&\left. \begin{aligned}
				&0 = 2\lambda_1(x_1-x_2) + 2 \lambda_2 x_1 - \dot u_1 \\
				&0 = 2y_1(\lambda_1-\lambda_2) - y - \dot v_1 \\
				&0 =  2\lambda_2 (x_2-x_1) - \dot u_2
			\end{aligned} \qquad \right\} &&\textrm{: additional algebraic constraint} \\
			&\left. \begin{aligned}
				&0 = x_1^2 + y_1^2 - 1 \\
				&0 = (x_1-x_2)^2 + y_1^2 - 1
			\end{aligned} \hspace{18.5mm} \right\} &&\textrm{: original algebraic constraint}
		\end{aligned} \right. 
	\end{equation}
	In order to reduce the index we can so differentiate in time the algebraic constraints: starting off with the newly added one what we obtain is (mathematical simplification are already performed)
	\begin{align*}
		\frac d{dt} \big( 2\lambda_1(x_1-x_2) + 2 \lambda_2 x_1 - \dot u_1 \big) & = 2 \lambda_1' (x_1-x_2) + 2\lambda_1(u_1 - u_2) + 2 \lambda_2' x_1 + 2\lambda_2 u_1 + \dot u_1' \\
		\frac d{dt} \big( 2y_1(\lambda_1-\lambda_2) - y - \dot v_1 \big) & = 2v_1 (\lambda_1 - \lambda_2) +2 y_1(\lambda_1'-\lambda_2') - \dot v_1' \\
		\frac d{dt} \big( 2\lambda_2 (x_2-x_1) - \dot u_2 \big) & = 2\lambda_2' (x_2 - x_1) + 2 \lambda_2(u_2 - u_1) - \dot u_2'
	\end{align*}
	In this case the differentiation already presents differential terms (in the variables $\dot u_1', \dot v_1', \dot u_2', \lambda_1', \lambda_2'$) and so \textit{it doesn't make sense} to continue with the differentiation in time of this expressions. Differentiating instead the \textit{original} algebraic constraints what we obtain is
	\begin{align*}
		\frac d{dt} \big( x_1^2 + y_1^2 \big) & = 2x_1 x_1' + 2 y_1 y_1' = 2x_1 u_1 + 2 y_1 v_1 && (a) \\
		\frac d{dt} \big( (x_1-x_2)^2 + y_1^2 - 1 \big) & = 2(x_1-x_2)(x_1'-x_2') +2 y_1 y_1' = 2(x_1-x_2)(u_1-u_2) + 2y_1 v_1 &&(b)
	\end{align*}
	This equations (after substituting all the variables $x_i' = u_i$) presents no differential part, hence we have to reduce one more time the index by differentiating only this two algebraic equations in time:
	\begin{align*}
		\frac d{dt}(a) & = 2 u_1^2 + 2x_1 \dot u_1 + 2 v_1^2 + 2 y_1 \dot v_1 && (c)\\
		\frac d{dt}(b) & = 2 (u_1 - u_2)^2 + 2 (x_1 - x_2)(\dot u_1 - \dot u_2) +2 v_1^2 +2 y_1 \dot v_1  && (d) \\
	\end{align*}
	We still need to reduce the index (because this expression are still algebraic) and after this more differentiation we finally obtain an ordinary differential equation:
	\begin{align*}
		\frac d{dt}(c) & = 6 u_1 \dot u_1 + 6 v_1\dot v_1 + 2 x_1 \dot u_1' + 2 y_1 \dot v_1' \\
		\frac d{dt}(d) & = 6(u_1-u_2)(\dot u_1 - \dot u_2) + 6 v_1\dot v_1 + 2 (x_1 -x_2)(\dot u_1' - \dot u_2') + 2 y_1 \dot v_1' \\
	\end{align*}
	
	Ended the index reduction used to transform the algebraic constraints in ordinary differential part, we can rewrite such obtained ODEs in a matrix form as
	\[ \underbrace{ \matrix{ 2(x_1-x_2) & 2x_1 & - 1 & 0 & 0 \\
	2y_1 & -2y_1 & 0 & - 1 & 0 \\
	0 & -2(x_1-x_2) & 0 & 0 & -1 \\ 
	0 & 0 & 2x_1 & 2y_1 & 0 \\
	0 & 0 & 2(x_1 x_2) & 2y_1 & - 2(x_1 - x_2) } }_{\mat E} \underbrace{\vector{\lambda_1' \\ \lambda_2' \\ \dot u_1' \\ \dot v_1' \\ \dot u_2'}}_{\vett z'} = \underbrace{\vector{2\lambda_2 (u_2-u_1) - 2 \lambda_2 u_1 \\
	2v_1(\lambda_2 - \lambda_1) \\ 2 \lambda_2 (u_1- u_2) \\ -6u_1 \dot u_1 - 6v_1 \dot v_1 \\ - 6(u_1-u_2)(\dot u_1 - \dot u_2) - 6 v_1 \dot v_1} }_{\vett G} \]
	After all this steps we can so consider that the equivalent ordinary differential system of the initial DAE problem reported in equation \ref{eq:dae:dummyinitial} is the one determined as
	\begin{equation}
	\begin{cases}
		x_1' = u_1 \\ y_1' = v_1 \\ x_2' = u_2 \\ u_1' = \dot u_1 \\ v_1' = \dot v_1 \\ u_2' = \dot u_2 \\
		\mat E \vett z' = \vett G
	\end{cases}
	\end{equation}
	The initial condition of such ordinary differential equation satisfy the hidden \textit{original} constraints
	\[ 0 = x_1^2 + y_1^2 - 1 \qquad 0 = (x_1-x_2)^2 + y_1^2 - 1 \]
	\[ 0 = 2\lambda_1(x_1-x_2) + 2 \lambda_2 x_1 - \dot u_1 \qquad 0 = 2y_1(\lambda_1-\lambda_2) - y - \dot v_1 \qquad 0 =  2\lambda_2 (x_2-x_1) - \dot u_2  \]
	but also the expression retrieved while performing the index reduction on the system $(a),(b),(c),(d)$, so
	\[ 2x_1u_1 + 2 y_1 v_1 = 0 \qquad 2(x_1-x_2)(u_1-u_2) + 2 y_1 v_1 = 0 \]
	\[ 2 u_1^2 + 2x_1 \dot u_1 + 2 v_1^2 + 2 y_1 \dot v_1 = 0 \qquad 2 (u_1 - u_2)^2 + 2 (x_1 - x_2)(\dot u_1 - \dot u_2) +2 v_1^2 +2 y_1 \dot v_1 = 0   \]

\subsection{Kernel computation and index reduction}
	Given a differential algebraic equation $\mat E(t,\vett y) \vett y' = \vett G(t,\vett y)$, if the matrix $\mat E$ is singular then it means that exists at least one vector $\vett v \neq 0$ such that $\mat E \vett v = 0$. This operation is indeed the \textbf{kernel computation} of the matrix $\mat E$. We define the \de{kernel} of such matrix the sub-space described by the set
	\[ \ker\{\mat E\} = \big\{ \vett v \textrm{ such that } \mat E \vett v = 0 \} = \textrm{span}\{\vett v_1, \dots , \vett v_k\}  \]
	where the vectors $\vett v_1,\dots, \vett v_k$ are determining a basis of $\ker\{\mat E\}$. Computing the kernel of $\mat E^T$ evaluates to $\ker\{\mat E^T\} = \{ \vett w \ | \ \vett w^T \mat E = 0 \} = \textrm{span}\{\vett w_1^T,\dots, \vett w_p^T\}$. Building so the matrix $\mat K = \matrix{\vett w_1 & \dots & \vett w_p}$ (considering all the vectors of the basis of $\ker\{\mat E^T\}$) as the \textit{concatenations of the vectors $\vett w_i$} implies that
	\[ \mat K^T \mat E = \matrix{ \vett w_1^T \mat E \\ \vdots \\ \vett w_p^T \mat E } = \vector{0 \\ \vdots  \\ 0} \]
	
	The main idea is that if we are indeed able to compute such matrix $\mat K$ associated to the kernel of $\mat E^T$, then we have a way to easily compute the invariant of the differential algebraic equation, in fact 
	\[ \mat K^T(t,\vett y)\Big(\mat E(t,\vett y) \vett y' = \vett G(t,\vett y) \Big) \qquad \Rightarrow \qquad \vett 0 = \mat K^T(t, \vett y)  \vett G(t,\vett y)\]
	
	In general $\mat K \in \mathds R^{n\times p}$ is a rectangular matrix (where $p < n$) and so if we can determine a matrix $\mat L \in \mathds R^{n\times (p-n)}$ such that $\mat M = \matrix{\mat K & \mat L} \in \mathds R^{n\times n}$ is non-singular, then we can observe that we can separate the algebraic part of the DAE from the differential part, in fact
	\begin{align*}
		\mat M^T\Big(  \mat E \vett y' & = \vett G \Big) \\
		\mat M^T \mat E \vett y' & = \mat M^T \vett G \qquad \Rightarrow \begin{cases}
			\cancel{\mat K^T \mat E \vett y'} = \mat K^T \vett G \\ \mat L^T \mat E \vett y' = \mat L^T \vett G
		\end{cases}
	\end{align*}
	that can be reduced to a form
	\begin{equation}
	\begin{cases}
		\tilde{\mat E}_1 \vett y' = \tilde{\vett G}_1  \qquad & \textrm{differential part}\\ 0 = \tilde{\vett G}_2 & \textrm{algebraic part}
	\end{cases}
	\end{equation}
	We have so separated the algebraic part from the differential one, so we can differentiate this second one in order to reduce the index and obtain a system in the form
	\[ \begin{cases}
		\tilde{\mat E}_1 \vett y' = \tilde{\vett G}_1 \\
		\tilde{\mat E}_1 \vett y' = \tilde{\vett G}_1  
		-\pd{\tilde{\vett G}_2}{\vett y} \vett y' = \pd{\tilde{\vett G}_2}{t} 
	\end{cases} \]
	This newly constructed differential algebraic problem can be reduce following the same procedure until when the kernel of $\mat E^T$ has a null dimension (it is composed only by the zero vector).
	
	\paragraph{Computation of the kernel} With this premise being said, we have to find a way to compute the kernel of the square matrix $\mat E$; this can be achieved, as example, using the LU decomposition. Considering in fact the decomposition $\mat{PEQ} = \mat {LU}$ (where $\mat P, \mat Q$ are permutation matrices), as was previously discussed, can be inverted to a form $\mat E = \mat P^{-1} \mat{LUQ}^{-1} = \mat P^{-1} \mat L M$ where $\mat M$ can has the \textit{upper rectangle} that's non-null and the lower one that's null (so is in the form $\mat M = \matrix{\mat M_1 \\ 0}$); knowing that $\mat L$ is non-singular (for the definition of the LU decomposition), then also $\mat P^{-1} \mat L$ is nonsingular and so we can build the matrix
	\[ \mat M = \mat L^{-1} \mat{PE} = \matrix{\mat M_1 \\ 0} \]
	Multiplying by appropriately matrices of the form $\matrix{\I & 0 }$ and $\matrix{0 & \I}$ this expressions we can observe that 
	\begin{align*}
		\underbrace{\matrix{\I & 0 } \mat L^{-1} \mat P}_{=\mat R^T} \mat E = \matrix{\I & 0} \matrix{\mat M_1 \\ 0} & = \mat M_1 \neq 0 \\
		\underbrace{\matrix{0 & \I } \mat L^{-1} \mat  P}_{=\mat K^T}  \mat E = \matrix{0 & \I } \matrix{\mat M_1 \\ 0} & = 0 
	\end{align*}
	This method allowed us to compute the required matrices as $\mat K = \mat P^T \mat L^{-T} \matrix{0 \\ \I}$ and $\mat R = \mat P^T \mat L^{-T} \matrix{\mat I \\ 0}$ that \textit{stacked} determined the non-singular matrix $\matrix{\mat K & \mat R} = \mat P^T \mat L^{-T} \matrix{0 & \I \\ \I & 0}$.
	
\section{Semi-explicit form}
	Given a differential algebraic equation in the form $\vett F(\vett z,\vett z', t) = 0$ that can be expressed as
	\[ \begin{cases}
		\vett x' & = \vett f(\vett x, \vett y, t) \\ 0 & = \vett g(\vett x, \vett y, t)
	\end{cases} \]
	where $\vett x$ and $\vett y$ are respectively the differential and algebraic variables of the problem and are such that $\vett z = (\vett x , \vett y)$. Let us so consider $x \subseteq z$ the variables in $\vett F(\vett z,\vett z', t)$ that are appearing as derivative, then we can defined $\vett x'$ as $\dvett x$ thus
	\[ \vett F(\vett z, \vett z', t) = \vett F(\vett z, \dvett x, t) = \vett F (\vett x, \vett y, \dvett x, t) = 0 \]
	This can be rewritten as
	\[ \begin{cases}
		\vett x' & = \dvett x \\
		0 & = \vett F(\vett x, \vett y, \dvett x, t)
	\end{cases} \]
	Moreover calling $\vett w = (\dvett x, \vett y)$ what we obtain is a \textbf{differential algebraic equation} in the so called \de{semi-explicit form}:
	\begin{equation}
	\begin{cases}
		\vett x & = \dvett x = \vett H(\vett x, \vett w, \vett t) \\
		0 & = \vett G(\vett x, \vett w, t)
	\end{cases}
	\end{equation}

	\paragraph{Semi-explicit DAE of index 1} Considering a DAE in the form
	\[ \begin{cases}
		\vett x' & = \vett f(\vett x, \vett y, t) \\ 0 & = \vett g(\vett x, \vett y, t)
	\end{cases} \]
	then if it is possible to solve $\vett g$ to obtain $\vett y$ so in the form $\vett y = \vett H(\vett x, t)$, then what we have is that $\vett g\big(\vett x, \vett H(\vett x, t), t\big) = 0$ for all values $\vett x, t$, and so we can consider the DAE as
	\[ \begin{cases}
		\vett x' & = \vett f(\vett x, \vett y, t) \\ \vett y & = \vett H(\vett x, t)
	\end{cases} \] 
	Differentiating the second equation (the algebraic part) in time determines
	\[ \vett y ' = \frac{d\vett H (\vett x, t)}{dt} = \pd{\vett H(\vett x, t)}{\vett x} \vett x' + \pd{\vett H(\vett x,t)}{t} = \pd{\vett H(\vett x, t)}{\vett x} \vett f(\vett x, \vett y, t) + \pd{\vett H(\vett x,t)}{t} = \vett L(\vett x, \vett y, t)  \]
	and so the differential algebraic equation after one differentiation (so with a index reduction) becomes
	\[ \begin{cases}
		\vett x' & = \vett f(\vett x, \vett y, t) \\ \vett y' & = \vett L(\vett x, \vett y, t)
	\end{cases} \] 
	In general an ODE in semi-explicit form of index 1 can be solved without performing the derivation by simply considering
	\begin{equation} \label{eq:dae:semiexplicitsubs}
		\vett x' = \vett f(\vett x, \vett y, t) = \vett f\big(\vett x, \vett H(\vett x, t), t\big) = \tilde{\vett f}(\vett x, t)
	\end{equation}
\subsection{Implicit function theorem}
	Equation \ref{eq:dae:semiexplicitsubs} relies on the fact that from $\vett g ( \vett x, \vett y, t) = 0$ allows us to extract the map $\vett H(\vett x,t)$ that allows to explicitly compute $\vett y$ (so satisfying $\vett g\big(\vett x, \vett H(\vett x, t), t\big) = 0$), however up to now we cannot be sure that such map $\vett H$ really exists.
	
	\paragraph{Simplified version} Considering a function $f(x,y) \in \mathcal C^1(\mathds R^2)$ and a point $(x_0,y_0)$ such that $f(x_0,y_0) = 0$ and $\pd f y(x_0,y_0) \neq 0$, then there exists an open interval $(x_0-\delta, x_0+\delta)$ such that
	\[ y = H(x) \qquad \textrm{and} \qquad f\big(x, H(x)\big) = 0 \quad \forall x \in (x_0-\delta, x_0 +\delta) \]
	Moreover we have that
	\begin{equation}
		\frac{dH(x)}{dx} = - \left(\pd{f(x,y)}{y}\right)^{-1} \pd{f(x,y)}{x}
	\end{equation}
	
	Considering the simple example of the function $f(x,y) = y^2 - x$, the point $(x_0,y_0) = (1,1)$ satisfies the conditions of the implicit function theorem, in fact
	\[ f(x_0,y_0) = 0 \qquad \textrm{and} \qquad \left.\pd{f}{y}\right|_{x = x_0,y = y_0} = 2y \Big|_{x = 1,y = 1} = 2 \neq 0 \]
	then we are sure that there exists a matrix $H$ that allows to express $y$ as function of $x$, in fact rewriting $y^2 = x$ and so $y = \pm \sqrt x$. The choice of the sign strictly depends on the value of the point $(x_0,y_0)$: in this case we have that $y = H(x) = \sqrt x$. In contrary, if we would have chosen the initial point $(x_0,y_0) = (1,-1)$ the condition for the theorem would have been still verified but the resulting map would have been $y = - \sqrt x$.\\
	Choosing instead the point $(x_0,y_0) = (0,0)$ would have resulted in the violation of the condition $\pd f y \neq 0$, thus the theorem cannot be applied: in such point we can't in fact discriminate the sign of the map $y = \pm \sqrt x$. However by inverting $x$ and $y$ in the definition, we can define a map $H(y)$ that determines $x$, in fact
	\[ f(0,0) = 0 \qquad \text{and} \qquad \left.\pd{f}{x}\right|_{x = x_0,y = y_0} = 1 \neq 0 \]
	and so $x = H(y) = y^2$.
	
	\paragraph{Implicit function theorem} Given a function $\vett f : \mathcal A \subseteq \mathds R^{n+m} \rightarrow \mathds R^m$ (where $n,m$ are respectively the number of \textit{independent} and \textit{dependent} variables) and a point $(\vett x_0,\vett y_0) \in \mathcal A$ (with $\vett x_0 \in \mathds R^n,\vett y_0\in \mathds R^m$) such that $\vett f(\vett x_0,\vett y_0) = 0$ and the matrix $\pd{\vett f}{\vett y}(\vett x_0,\vett y_0)$ is non-singular, then there exists two open sets $\mathcal U \subseteq \mathds R^n,\mathcal V \subseteq \mathds R^m$ on top of which is defined a map $\vett \phi: \mathcal U \rightarrow \mathcal V$, $\vett \phi \in \mathcal C^1(\mathcal U,\mathcal V)$, such that
	\[ \vett y_0 = \vett \phi(\vett x_0) \qquad \Leftrightarrow \qquad \vett f\big(\vett x_0, \vett \phi (\vett x_0)\big) = 0 \quad \forall \vett x \in \mathcal U, \vett \phi(\vett x) \in \mathcal V \]
	Moreover
	\begin{equation}
		\pd{\vett \phi}{\vett x} = - \left[ \pd{\vett f\big(\vett x,\vett \phi(\vett x)\big)}{\vett x} \right]^{-1} \pd{\vett f\big(\vett x,\vett \phi(\vett x)\big)}{\vett x}
	\end{equation}
	
	\paragraph{General form} The same theorem can be stated similarly considering an initial function $\vett \phi(\vett z): \mathds R^n \rightarrow \mathds R^m$ where $n>m$. The vector $\vett z = (x_1, x_2,\dots, x_n)$ can be partitioned, upon reordering of the variables, in 2 sets $\vett z = (\vett x, \vett y)$ (where $\vett x \in \mathds R^{n-m}$ contains the independent variables while $\vett y \in \mathds R^m$ the dependent one) in such a way that the matrix $\pd{\vett \phi}{\vett y}(\vett x_0,\vett y_0)$ is non-singular: this so allows to use the implicit function theorem to determine a map $\vett \psi$ such that $\vett y = \vett \psi(\vett x)$.
	
	We can observe in fact that if the matrix $\pd{\vett \phi}{\vett z} \in \mathds R^{m\times n}$ is full rank (hence all rows of the jacobian are linearly independent), then there exists $m$ linearly independent columns while the remaining $n-m$ are linearly dependent from the others: reordering so the independent/dependent variables allows us to distinguish $\vett x$ from $\vett y$. \\
	For this reason the map $\vett \psi$ is in general not unique because there exists multiple partitioned sets $(\vett x,\vett y)$ determining a non-singular jacobian.
	
	\paragraph{Semi-explicit DAE} Recalling a differential algebraic equation in semi-explicit form 
	\[\begin{cases}
		\vett x' & = \vett f(\vett x, \vett y, t) \\ 0 & = \vett g(\vett x, \vett y, t)
	\end{cases} \]
	where $\vett x\in \mathds R^n$ are the so called \textit{differential states}, $\vett y \in \mathds R^m$ the \textit{algebraic states} and the maps are $\vett f: \in \mathds R^{n+m+1}\rightarrow \mathds R^n$ and $\vett g: \in \mathds R^{n+m+1}\rightarrow \mathds R^m$. If the matrix $\pd{\vett f}{\vett y} \in \mathds R^{m\times m}$ is non-singular, then for the implicit function theorem we can define the independent and dependent variables respectively as $\vett z^i = (\vett x, t), \vett z^d = \vett y$ such that exists a map $\vett \psi$ for which $\vett y = \vett \psi(\vett x, t)$ and
	\[ \vett y' = \pd{\vett \psi}{\vett x} \vett x' + \pd{\vett \psi}{t} = \pd{\vett \psi}{\vett x} f + \pd{\vett \psi}{t} = \vett h(\vett x, \vett y, t) \]
	This means that after 1 derivation the initial DAE in explicit form can be regarded as following ODE:
	\[\begin{cases}
		\vett x' & = \vett f(\vett x, \vett y, t) \\ \vett y' & = \vett h(\vett x, \vett y, t)
	\end{cases}  \qquad + \quad \vett g(\vett x, \vett y, t) = 0 \textrm{ : hidden invariant}\]
	
	
	
	
	
	
	
	