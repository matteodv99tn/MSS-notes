\chapter{Laplace Transform}
	
	The \de{Laplace Transform  $\L$} is a powerful operator that allow to express a function $f(t)$ in the domain of the time $t$ to a function $\hat f(s)$ expressed in the domain of the \textbf{complex variable} $s$; in a mathematical way the passage from one domain (in this case time) to another (complex variable) is expressed as 
	\[ f(t) \mapsto \hat f(s) = \laplace{f(t)} \]
	
	Not all function $f$ can be transformed, and this is due to the existence (or not) of the following integral that is used to calculate the transform of the function:
	\begin{equation}
		\hat f(s) = \int_{0^-}^\infty f(t) e^{-st} dt = \lim_{\varepsilon\rightarrow 0^+} \lim_{M\rightarrow \infty} \int_{-\varepsilon}^M f(t) e^{-st}\, dt
	\end{equation}
	
	This mathematical tool is very powerful because it can transform \textbf{differential equation} (in the domain of the time) \textbf{into algebraic equation} (in the domain of $s$) which are much easier to solve.
	
	\begin{note}
		This concept can be seen with a logarithm analogy: the product of two number, by the logarithm rule, is easier to calculate because
		\[ a\cdot b \mapsto \log a + \log b \]
		so by this with the \textbf{logarithm} you can convert \textbf{product} into \textbf{sums} that are easier to manipulate.
	\end{note}

	In mechanical system is often required to solve linear differential equation in order to describe the time response of the system itself: this can be done with analytical techniques (such the \textit{constant variations} method), but can be very tricky to solve, or by using the Laplace transform as follows:
	\begin{itemize}
		\item with the \textbf{Laplace transform} the differential equation is converted into an algebraic one;
		\item by analyzing this equation you can determine the \textbf{frequency response} of the system object of study;
		\item with the \textbf{Laplace inverse transform} it's possible to re-convert the solution from the domain of the complex variable $s$ into the domain of time $t$. 
	\end{itemize}

\section{Transform properties}
	
	The Laplace transform has some important properties that can simplify the hand-made calculus operation; the first important thing to keep is mind is that the Laplace transform is a \de{linear operator}, so for all functions $f(t)\mapsto \hat f(s)$ and $g(t)\mapsto \hat g(s)$ and real constants $a,b\in \mathds R$ it's true that
	\begin{equation}
		h(t) := a \, f(t) + b\, g(t) \mapsto \hat h(s) := a\, \hat f(s) + b \, \hat g(s)
	\end{equation} 
	
	\begin{demonstration} \label{lap:dem:linearity}
		The linearity property can be demonstrated by applying the Laplace transform equation to the linear combination of two function:
		\begin{align*}
			\laplace{a \, f(t) + b\, g(t)} & = \int_{0^-}^\infty \Big( a\, f(t) + b\, g(t)\Big)e^{-st}\, dt \\
			& = a \lapint f(t) e^{-st}\, dt + b \lapint g(t) e^{-st}\, dt \\
			& = a \, \hat f(s) + b \, \hat g(s)
		\end{align*}
		Note that in order to prove this property we had to use the property of the integral that allowed us to split him in two separate integrals: this can be true in a general case but some function might not satisfy this option, like it can be seen in the example \ref{lap:ex:integrationproblem}.
	\end{demonstration}
	
	\begin{example}{: integrable and non-integrable function} \label{lap:ex:integrationproblem}
		Consider the piecewise defined function (dependent from the parameter $n$) for $t\geq 0$
		\[ f_n(t) = \begin{cases}
			nt & 0 \leq t \leq 1/n \\
			2-nt \qquad & 1/n \leq t \leq 2/n \\ 0 & \textrm{otherwise}
		\end{cases}\]
		By pushing the value $n$ to infinity we can see that the function $f(t)=\lim_{n\rightarrow \infty} f_n(t)$ has always a value of zero $\forall t$; by computing the integral (from zero to infinity) it's possible to calculate the area generated by this function that's equal to
		\[ F_n(t) = \int_0^\infty f_n(t)\, dt = \frac 1 n \qquad \xrightarrow{n\rightarrow \infty} \quad 0\]
		So the area associated to this function, if pushing $n\rightarrow \infty$, is zero, as expected because the function is always zero in it's domain, so this might give the (bad) idea that's possible to take out the limit of the integral outside, in the sense that
		\[  F(t) = \int_0^\infty \lim_{n\rightarrow \infty} f_n(t) \, dt = \lim_{n\rightarrow \infty} \int_0^\infty f_n(t)\, dt\]
		
		\vspace{3mm}
		Considering now another piecewise function $g_n$ defined for $t\geq 0$
		\[ g_n(t) = \begin{cases}
			n^2t & 0 \leq t \leq 1/n \\
			2n-n^2t \qquad & 1/n \leq t \leq 2/n \\ 0 & \textrm{otherwise}
		\end{cases}\]
		As in the previous case by computing the limit we see that the function $g = \lim_{n\rightarrow \infty} g_n$ is always zero in it's domain; now if we consider the position of the limit we can see that the result of the computed area differs, in fact
		\[ \lim_{n\rightarrow\infty} \int_0^\infty g_n(t)\, dt = \frac{\frac 2 n n}{2} = 1 \qquad \neq \qquad \int_0^\infty \lim_{n\rightarrow \infty} g_n(t)\, dt = \int_0^\infty 0\, dt = 0 \]
		So in a general we have to keep in mind that
		\[ \lim \int \quad \neq \quad \int\lim \]
		This concept, for example, must be kept in mind when applying the linearity rule (or in general any other properties).
	\end{example}

	
	Another important fact is associated to the \de{scale change} of the time axes; in particular by stretching/expanding the time axes by a value $a>0$ of a function $f(t) \mapsto \hat f(s)$ it's true that:
	\begin{equation}
		f(at) \mapsto \frac 1 a \, \hat f\left(\frac s a\right)
	\end{equation}

	\begin{demonstration}
		As in demonstration \ref{lap:dem:linearity}, the property of the scale change can be verified by using the definition of the Laplace transform using the change of variables $at = z$ (that means $t= z/a$):
		\begin{align*}
			\laplace{f(at)} & = \int_{0^-}^\infty f(at)\, dt \\
			& = \lapint f(z) e^{-sz/a} \, \frac{dz}{a} \\
			& = \frac 1 a \, \hat f\left(\frac s a\right)
		\end{align*}
	\end{demonstration}
		
	Other two important properties are related to the \de{translation} in respect to the $s$ axes as in respect to the $t$ axes (by a coefficient $a>0$) by using the relation that follows:
	\begin{equation}
		e^{at} f(t) \mapsto \hat f(s-a) \qquad \, \qquad f(t-a) \mapsto e^{-at} \hat f(s)
	\end{equation}

	\begin{demonstration}
		The property of the translation in respect of the $s$ complex variable is done as follows:
		\begin{align*}
			\laplace{e^{at}f(t)} & = \lapint e^{at} f(t) e^{-st} \, dt = \lapint f(t) e^{(a-s)t} \, dt \\ & = \hat f(s-a)
		\end{align*}
		The demonstration of the translation in respect to time $t$ is a little bit longer and it involves the change of coordinates $z = t-a$:
		\begin{align*}
			\laplace{f(t-a)} & = \lapint f(t-a) e^{-st} \, dt = \lapint f(z) e^{-s(z-a)} \, dz \\ 
			& = e^{-sa} \lapint f(z) e^{-sz}\, dz \\ & = e^{-as} \hat f(s)
 		\end{align*} 
	\end{demonstration}
	
	\subsection*{Existence of the transform}
		Note that not all function can be transformed using the Laplace operator $\L$; if we choose the function $f(t) = e^{t^2}$, by applying the Laplace transform to this expression we fet the integral
		\[\laplace{e^{t^2}} = \lapint e ^{t^2-st} \, dt = \int_{0^-}^T e^{(t-s)t}\, dt + \int_T^\infty e^{(t-s)t}\, dt  \]
		If we consider a costant $T > \re{s}$ we can notice that the second integral $\int_T^\infty e^{(t-s)t}\, dt$ is not convergent for any variable $s\in\mathds C$.
		
		\vspace{3mm}
		In general if a continuous \textbf{function} $f(t)$ is \textbf{upper bounded}, after the time $t\geq T$, with an exponential order function described by the inequality $|f(t)| \leq M e^{Nt}$, then the \textbf{Laplace transform $\hat f(s)$ exists}. \\
		The existence of the transform is related to the convergence (or divergence) of the second of the following integrals:
		\[ \laplace f = \int_{0^-}^T f(t)e^{-st} \, dt + \int_T^\infty f(t)e^{-st} \, dt \]
		Now it $f$ respect the condition to be upper bounded that mean that also the following inequalities are respected:
		\begin{align*}
			\left| \int_T^\infty f(t)e^{-st}\, dt  \right| \leq  \int_T^\infty \left|f(t)e^{-st}\right|\, dt   \leq \int_T^\infty M e^{Nt} \left|e^{-st}\right|\, dt
		\end{align*}
		By evaluating the second integral we get it's value for which we can confirm the convergence of the integral when $\re{s} > N$:
		\[ = \int_T^\infty Me^{Nt} e^{-\re{s} t} \, dt = M \int_T^\infty e^{\left[N-\re{s}\right]t} \, dt\]
		\[ \xrightarrow{\re s > N} \qquad \lim_{T\rightarrow \infty} \int_T^{\infty} e^{\left[ N -\re s\right]t} \, dt = 0 \]
		
		
		
		
		
		
		
		
		
		
		
		
		
		
		
		
		
		
		
		
		
		
		
		
		
		
		
		
		
		
		
		
		
		
		
		
		
		
		