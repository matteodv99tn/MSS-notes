
\chapter{Appendix}
\section{Properties of the Laplace transform and transforms of common functions}
	
	\begin{table}[bht]
	\centering
	\caption{useful properties of the Laplace transform and transform of common functions.} \label{app:lap:properties} \label{app:lap:transforms}
	\begin{tabular}{c |c |c}
		\# & $f(t)$ & $\laplace{f(t)}(s)$ \\ \hline
		1 & $a\, f(t) + b \, g(t)$ & $a \, \hat f(s) + b \, \hat g(s)$ \\ 
		2 & $f(at)$ & $\frac 1 a \hat f\left(\frac s a\right)$ \\
		3 & $e^{at}f(t)$ & $\hat f(s-a)$ \\
		4 & $f(t-a)$ & $e^{-as} \hat f(s)$ \\
		5 & $\int_0^t f(z)\, dz$ & $\frac 1 s \hat f(s)$ \\
		6 & $f'(t)$ & $s \hat f(s) - f(0^+)$ \\
		7 & $f''(t)$ & $s^2 \hat f(s) - f'(0^+) - s f(0^+)$ \\
		8 & $\frac {d^n}{dt^n}f(t)$ & $s^n \hat f(s) - \sum_{j=0}^{n-1} s^{n-j-1} f^{(j)}(0^+) $ \\
		9 & $t^n \, f(t)$ & $(-1)^n \frac{d^n}{ds^n} \hat f(s)$ \\
		10 & $\big(f\otimes g\big)(t)$ & $\hat f(s) \hat g(s)$ \\ \hline
		1 & $1$ & $\frac 1 s$ \\
		2 & $t$ & $\frac 1 {s^2}$ \\
		3 & $t^k$ & $\frac{k!}{s^{k+1}}$ \\
		4 & $a^{bt}$ & $\frac{1}{s - b\log a }$ \\
		5 & $e^{at} \cos(\omega t)$ & $\frac{s-a}{(s-a)^2 + \omega^2}$ \\
		6 & $e^{at} \sin(\omega t)$ & $\frac{\omega}{(s-a)^2 + \omega^2}$ \\
		7 & $e^{at} \cosh(\omega t)$ & $\frac{s-a}{(s-a)^2 - \omega^2}$ \\
		8 & $e^{at} \sinh(\omega t)$ & $\frac{\omega}{(s-a)^2 - \omega^2}$ \\
		9 & $e^{at} t^n$ & $\frac{n!}{(s-a)^{n+1}}$ \\
		10 & $e^{\alpha t} - e^{\beta t}$ & $\frac{\alpha - \beta}{(s- \alpha)(s-\beta)}$ 
		
	\end{tabular}
	\end{table}

\newpage
\section{Resume: minimization}
	Given the minimization problem of the form
	\begin{align*}
		\textrm{minimize:} \qquad & f(\vett x) \\
		\textrm{subject to:}\qquad  \ & h_k(\vett x) = 0 \qquad && k=1,\dots, m\\
		& g_k(\vett x) \geq 0  &&k = 1,\dots,p
	\end{align*}
	the solution using the KKT can be found by firstly constructing the lagrangian $\lag $ as
	\[ \lag(\vett x, \vett \lambda, \vett \mu) = \vett x - \sum_{i=1}^m \lambda_i h_i(\vett x) - \sum_{i=1}^p \mu_i g_i(\vett x) \]
	Candidates to be minimum point can be computed by using the first order necessary condition (stationarity of the point) that requires:	
	\begin{align*}
		\nabla_{\vett x} \lag \big(\vstar x,\vstar \lambda,\vstar \mu\big) & = \vett 0 \\
		h_k\big(\vstar x\big) & =  0 \qquad && k = 0,\dots, m\\
		g_k\big(\vstar x\big) & \geq 0 && k = 0,\dots, p \\
		\mu_k^* g_k\big(\vstar x\big) & = 0 && k = 0,\dots, p \\
		\mu_k^* & \geq 0 && k = 0,\dots,p
	\end{align*}
	Determined the candidates, we have to compute the kernel of the gradient of the qualified constraints $H$ (the set of linearly independent gradients of the active constraints) and verify if the hessian of the lagrangian respect to the variables $\vett x$ is (semi-)positive defined:
	\[ \vett z^t \nabla_{\vett x} \lag \, \vett z  \begin{cases}
		> 0 \qquad &:\textrm{sufficient condition} \\
		\geq 0 \qquad &:\textrm{necessary condition} 
	\end{cases} \qquad \textrm{with } \vett z \in \ker\{ \nabla H \} \]

\newpage
\section{Resume: functional minimization}
	The minimization of a functional $\fun F(x)$ is based on determining the functions $x$ that determines a null first variation $\delta \fun F$ (and for that reason the fundamental lemma of calculus of variation will be used).
	
	\paragraph{Variation of the lagrangian} Considering functionals $\F(x)$ that present integral relation with $x$, in the form
	\[ \fun F(x) = \int_a^b L\Big(x(t), x'(t), t\Big) \, dt \]
	then the related variation obtained with integration by part is of the form
	\[ \delta \fun F = \int_a^b \left( \pd L x - \frac d {dt} \pd L{x'} \right) \delta_x\, dt + \left[ \pd L {x'} \delta_x \right]_a^b \]
	
	Similarly if $x$ appears in the integral with a derivative up to the second order, then the following relation must be considered:
	\begin{align*}
		\fun F(x) & = \int_a^b L\Big(x(t), x'(t), x''(t), t \Big) \, dt \\
		\delta \fun F & = \left( \pd L x - \frac{d}{dt} \pd L {x'} + \frac{d^2}{dt^2} \pd L{x''} \right) \delta_x \, dt + \left[ \left( \pd L {x'} - \frac d{dt}\pd L{x''} \right) \delta_x \right]_a^b + \left[ -\pd L {x''} \delta_x' \right]_a^b
	\end{align*}
	

\newpage
\section{Resume: optimal control problem}
	Given the optimal control problem  with states $\vett x$ and controls $\vett u$ in the form
	\begin{align*}
		\textrm{minimize:} \qquad & \phi\big(\vett x(a), \vett x(b)\big) + \int_a^b L(\vett x, \vett u, t)\, dt \\
		\textrm{subejct to:} \qquad & \vett x' = \vett f(\vett x, \vett u, t) && \textrm{ODE}\\
		& \vett B\big(\vett x(a), \vett x(b)\big) = 0 && \textrm{boundary conditions}\\
		& \int_a^b \vett g(\vett x,\vett u, t)\, dt = \vett g_0 && \textrm{integral constraints} \\
		& u \in \mathcal U && \textrm{control domain}
	\end{align*}
	the solution can be obtained by firstly removing the integral constraints by replacing each one of them with new ordinary differential equation in the form $z_i' = g_i(\vett x,\vett u, t)$ and two other boundary condition in the form $g_i(a) = 0$ and $g_i(b) = g_{i,0}$. With that stated the every optimal control problem with integral constraint can be rewritten in the form
	\begin{align*}
		\textrm{minimize:} \qquad & \phi\big(\vett x(a), \vett x(b)\big) + \int_a^b L(\vett x, \vett u, t)\, dt \\
		\textrm{subejct to:} \qquad & \vett x' = \vett f(\vett x, \vett u, t) && \textrm{ODE}\\
		& \vett B\big(\vett x(a), \vett x(b)\big) = 0 && \textrm{boundary conditions} \\
		& u \in \mathcal U && \textrm{control domain}
	\end{align*}
	where the new added variables $\vett z$ associated to the integral constraints are condensed in the state variables $\vett x$.
	
	With that said we can compute the hamiltonian $\mathcal H$ and the utility function $\mathcal B$ of the problem as
	\begin{align*}
		\H(\vett x, \vett u, \vett \lambda,t) & = L(\vett x, \vett u, t) + \vett \lambda\cdot \vett f(\vett x, \vett u, t) \\
		\B\big(\vett x(a),\vett x(b),\vett \mu\big) & = \phi\big(\vett x(a), \vett x(b) \big) + \vett \mu \cdot \vett B\big(\vett x(a),\vett x(b)\big)  
	\end{align*}
	The resulting boundary valued problem of the optimal control problem is so
	\[
	\left\{ \begin{aligned}
		& \vett x' = \vett f(x,u,t) = \pd{\mathcal H}{\vett \lambda} && \textrm{: original ODE} \\
		&\vett \lambda' = - \pd{\mathcal H}{\vett x} && \textrm{: adjoint ODE - co-equations} \\
		&\vett B\big(\vett x(a),\vett x(b)\big) = 0 && \textrm{: original BC} \\
		&\left. \begin{aligned}
			\pd{\B}{\vett x_a} + \vett \lambda(a) \\
			\pd{\B}{\vett x_b} - \vett \lambda(b)
		\end{aligned} \qquad \right\} &&\textrm{: adjoint BC} \\
		& \vett u(t) = \underset{\overline{\vett u} \in \mathcal U}{\textrm{argmin}} \big\{ \H\big(\vett x, \overline{\vett u},\vett \lambda, t\big) \big\} && \textrm{: Pontryagin min principle} \\
		& \pd{\mathcal H}{\vett u} = 0 && \textrm{: control equation}			
	\end{aligned} \right. \]
	where the solution of the Pontryagin minimum principle is the parametric solution that minimize the terms $\H$ that contains only the control $\vett u$ respect to the controls bound $\mathcal U$. 
	
\chapter{Final revision}
\section{January 24, 2022}
	Domande:
	\begin{itemize}
		\item parte A, domanda 6: perché sono sbagliate $h(x,y) = 0$ e $g(x,y)\geq 0$? Rispetto alle first order necessary condition.
		
		\item parte B, domanda 4: non mi risulta la ODE $x''-2x+\lambda = 1$ ma$-2x'' +2x - \lambda + 2$
		
		\item parte B, domanda 5: non mi risulta la ODE $2x-y''=0$ ma $2x-z-y''=0$; per quanto riguarda le condizioni al contorno non mi risulta $y'(0) = y(0)$ ma $y'(0) = 0$ e neanche non mi torna $x'(0) = x(0)$;
		
		\item parte B, domanda 7: ho sbagliato io
		
		\item parte B, domanda 8: non mi tornano le adjoint boundary conditions, dovrebbe esserci solo $\lambda_2(1) = 0$ (e non $\lambda_3(0) = \lambda_3(1) = 0$)
	\end{itemize}

\subsection*{Part B, question 4}
	Given the problem
	\begin{align*}
		\textrm{minimize:} \qquad & \mathcal F(x) = x(0) + \int_0^1 \Big(x^2 + \big(x'-t\big)^2 \Big)\, dt \\
		\textrm{subject to:} \qquad & \int_0^1 x\, dt = 0 \\ & x(1) = 2
	\end{align*}
	the related lagrangian is build as
	\[ \lag(x,\lambda,\mu) = \int_0^1 \underbrace{\Big( x^2 + \big(x'-t\big)^2 - \lambda x\Big)}_{=L}\, dt + x(0) - \mu \big(x(1)-2\big) \]
	It's first variation can be regarded as
	\begin{align*}
		\delta \lag & = \int_0^1 \Big( \big(2x - \lambda\big) \delta_x - 2\big(x'-t\big) \delta_{x'} \Big) \, dt - \int_0^1 x\,\delta_\lambda\, dt + \delta_{x(0)} - \mu \delta_{x(1)} - \big(x(1)-2\big) \delta_\mu \\
		& = \int_0^1 \left( \pd L x - \frac d {dt} \pd L{x'}\right) \delta_x\, dt + \left[\pd L{x'} \delta_x\right]_0^1  - \int_0^1 x\,\delta_\lambda\, dt + \delta_{x(0)} - \mu \delta_{x(1)} - \big(x(1)-2\big) \delta_\mu 
	\end{align*}
	Considering that $\pd L x = 2x- \lambda$, $\pd L{x'} = 2(x'-t)$ and so $\frac d{dt}\pd L{x'} = 2x''- 2$ we can explicit the variation as
	\begin{align*}
		\delta \lag = &\ \int_0^1 \big( 2x - \lambda -2x''+2  \big) \delta_x \, dt + 2\big(x'(1) - 1\big) \delta_{x(1)} - 2 x'(0) \delta_{x(0)} - \int_0^1 x\,\delta_\lambda\, dt \\ & + \delta_{x(0)} - \mu \delta_{x(1)} - \big(x(1)-2\big) \delta_\mu
	\end{align*}
	The resulting boundary value problem, solution of the functional minimization, can be obtained by setting to zero the terms related to each variation $\delta_\cdot$:
	\[ \begin{cases}
		2x - \lambda - 2x'' + 2 = 0 & : \delta_x \\
		1-2x'(0) = 0 & : \delta_{x(0)} \\
		\cancel{2x'(1) -2 - \mu = 0 } & : \delta_{x(1)} \quad \textrm{ trivially solved} \\
		x(1) = 2 & :\delta_\mu \\
		\int_0^1 x\, dt = 0 &: \delta_\lambda
	\end{cases} \]

\subsection*{Part B, question 5}
	Given the problem
	\begin{align*}
		\textrm{minimize:} \qquad & \mathcal F(x,y,z) = z(0) + \int_0^1 \Big(x^2 + xz +z'^2 + x'y'\Big)\, dt \\
		\textrm{subject to:} \qquad & z(1) = 2 \qquad y(0) = 1
	\end{align*}
	the differential equations managing the boundary value problem are based on the integral part of $\mathcal F$, so on $L(x,y,z) = x^2 + xz + z'^2 + x'y'$. In particular we have
	\begin{align*}
		\pd L x - \frac d{dt}\pd L{x'} & = 2x - z - \frac d{dt}\big(y'\big)= && 2x - z - y'' = 0 \\
		\pd L y - \frac d{dt}\pd L{y'} & = 0 - \frac d{dt}\big(x'\big) =&& x'' = 0 \\
		\pd L z - \frac d{dt}\pd L{z'} & = x - \frac d{dt}\big(2z'\big) =&& x - 2 z'' = 0 \\
	\end{align*}
	To determine the other boundary condition is necessary to compute the first variation of the lagrangian $\lag(x,y,z,\mu_1,\mu_2) = \mathcal F - \mu_1(z(1)-2) - \mu_2(y(0)-1)$ that's
	\begin{align*}
		\delta \lag = & \ \int_0^1 \dots\, dt + \left[\pd L {x'} \delta_x\right]_0^1 + \left[\pd L {y'} \delta_y\right]_0^1 + \left[\pd L {z'} \delta_z\right]_0^1  \\ & + \delta_{z(0)} - \mu_1 \delta_{z(1)} - \big(z(1)-2\big) \delta_{\mu_1} - \mu_2 \delta_{y(0)} - \big(y(0) - 1\big) \delta_{\mu_2} \\
		= & \ \int_0^1 \dots \, dt + y'(1) \delta_{x(1)} - y'(0)\delta_{x(0)} + x'(1) \delta_{y(1)} - x'(0) \delta_{y(0)} + 2z'(1)\delta_{z(1)} - 2z'(0) \delta_{z(0)}\\ & + \delta_{z(0)} - \mu_1 \delta_{z(1)} - \big(z(1)-2\big) \delta_{\mu_1} - \mu_2 \delta_{y(0)} - \big(y(0) - 1\big) \delta_{\mu_2} \\ 
	\end{align*}
	The boundary condition of the boundary value problem can be so obtained by setting equal to zero the terms related to each variation at the extremas, so:
	\begin{align*}
		\delta_{x(0)}: \quad & y'(0) = 0 \qquad & \delta_{x(1)}:& \quad  y'(1) = 0\\
		\delta_{y(0)}: \quad & \cancel{x'(0)+\mu_2 = 0} \qquad & \delta_{y(1)}:& \quad  x'(1) = 0 \\
		\delta_{z(0)}: \quad & 1 - 2z'(0) = 0 \qquad & \delta_{z(1)}:& \quad  \cancel{2z'(1)-\mu_2 = 0 } \\
		\delta_{\mu_1}: \quad & z(1) - 2 = 0 \qquad & \delta_{\mu_2}:& \quad  y(0) - 1 = 0 
	\end{align*}


\subsection*{Part B, question 7}
	Given the optimal control problem
	\begin{align*}
		\textrm{minimize:} \qquad & x(0) + \int_0^1 xu\, dt \\
		\textrm{subject to:} \qquad & x' =  y- u \qquad y' = xu \\
		& x(1) = 2 \\
		& u(t) \in [-2,1]
	\end{align*}
	the associated hamiltonian $\H$ and utility function $\B$ are
	\begin{align*}
		\H\big(x,y,u,\lambda_1,\lambda_2\big) & = xu + \lambda_1\big(y-u\big) + \lambda_2 xu \\
		\B(x_0,x_1,\mu) & = x_0 + \mu \big(x_1-2\big)
	\end{align*}
	The co-equation are computed directly from the hamiltonian as
	\begin{align*}
		\lambda_1' & = - \pd \H x = - \big(u + \lambda_2 u\big) \\
		\lambda_2' & = - \pd \H y = - \lambda_1
	\end{align*}
	The control law can be obtained by the Pontryagin minimum principle, so determining the controls $u$ in it's domain that minimize the function $\tilde H = xu - \lambda_1 u + \lambda_2 xu$:
	\[ u(t) = \underset{\overline{ u} \in \mathcal [-2,1]}{\textrm{argmin}} \left\{ \overline u\big( x - \lambda_1 + \lambda_2 x\big) \right\} = \begin{cases}
		-2 \quad & \textrm{if } x - \lambda_1 + \lambda_2 x > 0 \\
		1 & \textrm{if } x - \lambda_1 + \lambda_2 x < 0 
	\end{cases} \]
	The adjoint boundary conditions are instead dependent on the utility function $\B$, in particular
	\begin{align*}
		\lambda_1(0) = -\pd \B {x_0} = - 1 \qquad \lambda_2(0) = - \pd \B{y_0} = 0 \qquad \lambda_2(1) = \pd \B{y_1} = 0
	\end{align*}


\subsection*{Part B, question 8}
	Given the optimal control problem
	\begin{align*}
		\textrm{minimize:} \qquad & x(1) + 2 y(0) + \int_0^1 \big(1+x+y\big) u^2\, dt \\
		\textrm{subject to:} \qquad & x' =  y \qquad y' = x+u \\
		& \int_0^1 \big(x+u^2\big)\, dt = 2 \\
		& x(0) = 1 \qquad x(1) = 2 \qquad y(0) = 2 
	\end{align*}	
	the first thing is to get rid of the integral constraint by adding a new state $z$ whose differential equation is $z' = x + u^2$ and bounded with $z(0) = 0$ and $z(1) = 2$:
	\begin{align*}
		\textrm{minimize:} \qquad & x(1) + 2 y(0) + \int_0^1 \big(1+x+y\big) u^2\, dt \\
		\textrm{subject to:} \qquad & x' =  y \qquad y' = x+u \qquad z' = x+u^2 \\
		& x(0) = 1 \qquad x(1) = 2 \qquad y(0) = 2  \qquad z(0) = 0 \qquad z(1) = 2
	\end{align*}	
	With that problem as here presented, the hamiltonian $\H$ and the utility function $\B$ are so
	\begin{align*}
		\H\big(x,y,z, u,\lambda_1,\lambda_2,\lambda_3\big)  = & \ \big(1+x+y\big)u^2 + \lambda_1 y + \lambda_2(x+u) + \lambda_3\big(x+u^2\big) \\
		\B\big(x_0,x_1,y_0,z_0,z_1,\mu_1,\mu_2, \mu_3, \mu_4, \mu_5\big)  = & \ x_1 + 2 y_0 + \mu_1\big(x_0-1\big) + \mu_2 \big(x_1-2\big) + \mu_3 \big(y_0-2\big) \\ & + \mu_4 z_0 + \mu_5 \big(z_1-2\big)
	\end{align*}
	Using the Pontryagin maximum principle, we know that the hamiltonian must be stationary respect to the chosen control so
	\[ \pd \H u = 2u(1+x+y) + \lambda_2 + 2 u \lambda_3 = 0  \]
	and inverting to explicit the control
	\[ u(t) = - \frac{\lambda_2(t)}{2 \big(1+x(t)+y(t) + \lambda_3(t)\big)} \]
	The adjoint equations are
	\begin{align*}
		\lambda_1' & = - \pd \H x = - \big( u^2 + \lambda_2 + \lambda_3 \big) \\
		\lambda_2' & = - \pd \H y = - \big( u^2 + \lambda_1 \big) \\
		\lambda_3' & = - \pd \H z = 0 
	\end{align*}
	with adjoint boundary conditions
	\begin{align*}
		& \cancel{\lambda_1(0) = - \pd \B{x_0} = -\mu_1} \qquad && \cancel{\lambda_1(1) =  \pd \B {x_1} =  1 + \mu_2} \\
		& \cancel{\lambda_2(0) = - \pd \B{y_0} = -2-\mu_3} \qquad && \lambda_2(1) =  \pd \B {y_1} = 0 \\
		& \cancel{\lambda_3(0) = - \pd \B{z_0} = -\mu_4} \qquad && \cancel{\lambda_3(1) =  \pd \B {z_1} = \mu_5} \\
	\end{align*}





	

	
	
	
	
	
	
	
	
	
	
	
	


